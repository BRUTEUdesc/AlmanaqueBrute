\chapter{Teórico}

\twocolumn

\section{Alguns Números Primos}

\subsection{Primo com Truncamento à Esquerda}
\large{Número primo tal que qualquer sufixo dele é um número primo} \\
\Large{357,686,312,646,216,567,629,137}

\subsection{Números Primos de Mersenne}
\large{Números primos da forma $2^m-1$}
\begin{table}[h]
  \centering
  \begin{tabular}{|c|c|}
    \hline
    \textbf{Expoente ($m$)} & \textbf{Representação Decimal} \\
    \hline
    \(2\) & 3 \\
    \hline
    \(3\) & 7 \\
    \hline
    \(5\) & 31 \\
    \hline
    \(7\) & 127 \\
    \hline
    \(13\) & 8,191 \\
    \hline
    \(17\) & 131,071 \\
    \hline
    \(19\) & 524,287 \\
    \hline
    \(31\) & 2,147,483,647 \\
    \hline
    \(61\) & $2,3 * 10^{18}$ \\
    \hline
    \(89\) & $6,1 * 10^{26}$ \\
    \hline
    \(107\) & $1,6 * 10^{32}$ \\
    \hline
    \(127\) & $1,7 * 10^{38}$ \\
    \hline
  \end{tabular}
\end{table}

\section{Constantes em C++}

\begin{center}
\begin{tabular}{|c|c|c|}
  \hline
  Constante & Nome em C++ & Valor \\
  \hline
  $\pi$ & \texttt{M\_PI} & 3.141592... \\
  \hline
  $\pi / 2$ & \texttt{M\_PI\_2} & 1.570796... \\
  \hline
  $\pi / 4$ & \texttt{M\_PI\_4} & 0.785398... \\
  \hline
  $1 / \pi$ & \texttt{M\_1\_PI} & 0.318309... \\
  \hline
  $2 / \pi$ & \texttt{M\_2\_PI} & 0.636619... \\
  \hline
  $2 / \sqrt{\pi}$ & \texttt{M\_2\_SQRTPI} & 1.128379... \\
  \hline
  $\sqrt{2}$ & \texttt{M\_SQRT2} & 1.414213... \\
  \hline
  $1 / \sqrt{2}$ & \texttt{M\_SQRT1\_2} & 0.707106... \\
  \hline
  $e$ & \texttt{M\_E} & 2.718281... \\
  \hline
  $\log_2{e}$ & \texttt{M\_LOG2E} & 1.442695... \\
  \hline
  $\log_{10}{e}$ & \texttt{M\_LOG10E} & 0.434294... \\
  \hline
  $\ln{2}$ & \texttt{M\_LN2} & 0.693147... \\
  \hline
  $\ln{10}$ & \texttt{M\_LN10} & 2.302585... \\
  \hline
\end{tabular}
\end{center}

\section{Operadores Lineares}

\subsection{Rotação no sentido anti-horário por $\theta ^\circ$}
\begin{equation*} 
\Large{
\begin{bmatrix}
    \cos \theta & -\sin \theta \\
    \sin \theta & \cos \theta
\end{bmatrix}
}
\end{equation*}

\subsection{Reflexão em relação à reta $y = mx$ }
\begin{equation*}
\LARGE{\frac{1}{m^2+1}}
\large{
\begin{bmatrix}
    1 - m^2 & 2m \\
    2m & m^2 - 1
\end{bmatrix}
}  
\end{equation*} 

\subsection{Inversa de uma matriz 2x2 A}
\begin{equation*} 
\large{
\begin{bmatrix}
    a & b \\
    c & d
\end{bmatrix}^{-1} =
\LARGE{\frac{1}{\text{det}(A)}}
\large{
\begin{bmatrix}
    d & -b \\
    -c & a
\end{bmatrix}
}
}
\end{equation*}

\subsection{Cisalhamento horizontal por K}
\begin{equation*} 
\Large{
\begin{bmatrix}
    1 & K \\
    0 & 1
\end{bmatrix}
}
\end{equation*}

\subsection{Cisalhamento vertical por K}
\begin{equation*} 
\Large{
\begin{bmatrix}
    1 & 0 \\
    K & 1
\end{bmatrix}
}
\end{equation*}

\subsection{Mudança de base}
\Large {$\vec{a}_\beta$ são as coordenadas do vetor $\vec{a}$ na base $\beta$.\\}
\Large {$\vec{a}$ são as coordenadas do vetor $\vec{a}$ na base canônica.\\}
\Large {$\vec{b1}$ e $\vec{b2}$ são os vetores de base para $\beta$.\\}
\Large {$C$ é uma matriz que muda da base $\beta$ para a base canônica.}
\begin{equation*}
\Large{C \vec{a}_\beta = \vec{a}}
\end{equation*} 
\begin{equation*}
\Large{C^{-1} \vec{a} = \vec{a}_\beta}
\end{equation*} 
\begin{equation*} 
\Large{
C = 
\begin{bmatrix}
    b1_x & b2_x \\
    b1_y & b2_y
\end{bmatrix}
}
\end{equation*}

\subsection{Propriedades das operações de matriz}
\begin{equation*} 
\large{(AB)^{-1} = A^{-1}B^{-1}}\\
\end{equation*}
\begin{equation*} 
\large{(AB)^{T} = B^{T}A^{T}}\\
\end{equation*}
\begin{equation*} 
\large{(A^{-1})^{T} = (A^{T})^{-1}}\\
\end{equation*}
\begin{equation*} 
\large{(A+B)^{T} = A^{T} + B^{T}}\\
\end{equation*}
\begin{equation*} 
\large{\text{det}(A) = \text{det}(A^{T})}\\
\end{equation*}
\begin{equation*} 
\large{\text{det}(AB) = \text{det}(A)\text{det}(B)}\\
\end{equation*}
\large{Seja $A$ uma matriz NxN:}
\begin{equation*} 
\large{\text{det}(kA) = K^N\text{det}(A)}\\
\end{equation*}