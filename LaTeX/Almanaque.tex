\documentclass[11pt, a4paper, oneside]{book}
\usepackage[T1]{fontenc}
\usepackage[utf8]{inputenc}
\usepackage{amssymb,amsmath}
\usepackage[brazil]{babel}
\usepackage[landscape, a4paper, total={6in, 8in}]{geometry}
\usepackage{comment}
\usepackage{datetime}
\usepackage[pdfusetitle]{hyperref}
\usepackage[all]{xy}
\usepackage{graphicx}
\addtolength{\parskip}{.5\baselineskip}
\setlength{\parindent}{0pt}

\usepackage{xcolor}
\usepackage{listings} %biblioteca pro codigo

\usepackage{multicol} %biblioteca pra duas colunas
\setlength{\columnseprule}{0.5pt}
\setlength{\columnsep}{10pt} %distancia do texto ate a barra
\def\columnseprulecolor{\color{black}}

% \usepackage{color}    %deixa o codigo colorido bonitinho
\geometry{
	a4paper,
	left=10mm,
	right=10mm,
	top=1in,
	bottom=1in,
}

\definecolor{gray}{rgb}{0.2, 0.2, 0.2} %cor pros comentarios
% \renewcommand{\footnotesize}{\small} %isso eh pra mudar o tamanho da fonte do codigo

\lstset{ %opcoes pro codigo
	language=C++,
	breaklines=true,
	keywordstyle=\color{black}\bfseries,
	commentstyle=\color{darkgray},
	breakatwhitespace=true,
	% frame=single,
	% numbers=left,
	% xleftmargin=0.50cm,
	rulecolor=\color{black},
	showstringspaces=false
	stringstyle=\color{purple},
	tabsize=2,
	basicstyle=\footnotesize,
	% basicstyle=\ttfamily\footnotesize, % fonte pro codigo, nao sei qual dos dois eu prefiro
}
\lstset{literate=
%   *{0}{{{\color{red!20!violet}0}}}1
%    {1}{{{\color{red!20!violet}1}}}1
%    {2}{{{\color{red!20!violet}2}}}1
%    {3}{{{\color{red!20!violet}3}}}1
%    {4}{{{\color{red!20!violet}4}}}1
%    {5}{{{\color{red!20!violet}5}}}1
%    {6}{{{\color{red!20!violet}6}}}1
%    {7}{{{\color{red!20!violet}7}}}1
%    {8}{{{\color{red!20!violet}8}}}1
%    {9}{{{\color{red!20!violet}9}}}1
%	 {l}{$\text{l}$}1
	{~}{$\sim$}{1} % ~ bonitinho
}

\title{BRUTE \\ UDESC}
\author{Eliton Machado da Silva, Enzo de Almeida Rodrigues, Eric Grochowicz, \\ João Vitor Frölich, João Marcos de Oliveira e Rafael Granza de Mello}	

\begin{document}

% \twocolumn
\date{\today}
\maketitle

\renewcommand{\contentsname}{Índice} %troca o nome do indice para indice

\tableofcontents
\newpage

%%%%%%
%
%
% Matemática
%
%
%%%%%%

\chapter{Matemática}

\section{Sum of floor(n div i)}


Computa $\sum_{i=1}^{n} \lfloor\frac{n}{i}\rfloor $



Computa o somatório de n dividido de 1 a n (divisão arredondado pra baixo).



\begin{itemize}
\item Complexidade de tempo: O( sqrt(n) ).
\end{itemize}

\hfill

\begin{multicols}{2}
\begin{lstlisting}[language=C++]
const int MOD = 1e9 + 7;

long long sumoffloor(long long n) {
    long long answer = 0, i;
    for (i = 1; i * i <= n; i++) {
        answer += n / i;
        answer %= MOD;
    }
    i--;
    for (int j = 1; n / (j + 1) >= i; j++) {
        answer += (((n / j - n / (j + 1)) % MOD) * j) % MOD;
        answer %= MOD;
    }
    return answer;
}
\end{lstlisting}
\end{multicols}

\hfill

\section{Eliminação Gaussiana}


Método de eliminação gaussiana para resolução de sistemas lineares.



\begin{itemize}
\item Complexidade de tempo: O(n³).
\end{itemize}





Dica: Se os valores forem apenas 0 e 1 o algoritmo [gauss\_mod2](gauss\_mod2.cpp) é muito mais rápido. 

\hfill

\begin{multicols}{2}
\begin{lstlisting}[language=C++]
const int N = 105;
const int INF = 2; // tanto faz

// n -> numero de equacoes, m -> numero de
// variaveis a[i][j] para j em [0, m - 1] ->
// coeficiente da variavel j na iesima equacao
// a[i][j] para j == m -> resultado da equacao da
// iesima linha ans -> bitset vazio, que retornara
// a solucao do sistema (caso exista)
int gauss(vector<bitset<N>> a, int n, int m, bitset<N> &ans) {
    vector<int> where(m, -1);

    for (int col = 0, row = 0; col < m && row < n; col++) {
        for (int i = row; i < n; i++) {
            if (a[i][col]) {
                swap(a[i], a[row]);
                break;
            }
        }
        if (!a[row][col]) {
            continue;
        }
        where[col] = row;

        for (int i = 0; i < n; i++) {
            if (i != row && a[i][col]) {
                a[i] ^= a[row];
            }
        }
        row++;
    }

    for (int i = 0; i < m; i++) {
        if (where[i] != -1) {
            ans[i] = a[where[i]][m] / a[where[i]][i];
        }
    }
    for (int i = 0; i < n; i++) {
        int sum = 0;
        for (int j = 0; j < m; j++) {
            sum += ans[j] * a[i][j];
        }
        if (abs(sum - a[i][m]) > 0) {
            return 0; // Sem solucao
        }
    }

    for (int i = 0; i < m; i++) {
        if (where[i] == -1) {
            return INF; // Infinitas solucoes
        }
    }
    return 1; // Unica solucao (retornada no
              // bitset ans)
}
\end{lstlisting}
\end{multicols}

\hfill

\begin{multicols}{2}
\begin{lstlisting}[language=C++]
const double EPS = 1e-9;
const int INF = 2; // it doesn't actually have to
                   // be infinity or a big number

int gauss(vector<vector<double>> a, vector<double> &ans) {
    int n = (int)a.size();
    int m = (int)a[0].size() - 1;

    vector<int> where(m, -1);
    for (int col = 0, row = 0; col < m && row < n; ++col) {
        int sel = row;
        for (int i = row; i < n; ++i) {
            if (abs(a[i][col]) > abs(a[sel][col])) {
                sel = i;
            }
        }
        if (abs(a[sel][col]) < EPS) {
            continue;
        }
        for (int i = col; i <= m; ++i) {
            swap(a[sel][i], a[row][i]);
        }
        where[col] = row;

        for (int i = 0; i < n; ++i) {
            if (i != row) {
                double c = a[i][col] / a[row][col];
                for (int j = col; j <= m; ++j) {
                    a[i][j] -= a[row][j] * c;
                }
            }
        }
        ++row;
    }

    ans.assign(m, 0);
    for (int i = 0; i < m; ++i) {
        if (where[i] != -1) {
            ans[i] = a[where[i]][m] / a[where[i]][i];
        }
    }
    for (int i = 0; i < n; ++i) {
        double sum = 0;
        for (int j = 0; j < m; ++j) {
            sum += ans[j] * a[i][j];
        }
        if (abs(sum - a[i][m]) > EPS) {
            return 0;
        }
    }

    for (int i = 0; i < m; ++i) {
        if (where[i] == -1) {
            return INF;
        }
    }
    return 1;
}
\end{lstlisting}
\end{multicols}

\hfill

\section{FFT}


Algoritmo que computa a transformada rápida de fourier para convolução de polinômios.



Computa convolução (multiplicação) de polinômios.



\begin{itemize}
\item Complexidade de tempo (caso médio): O(N * log(N))
\item Complexidade de tempo (considerando alto overhead): $O(n * log^2(n) * log(log(n)))$
\end{itemize}



Garante que não haja erro de precisão para polinômios com grau até $3 * 10^5$ e constantes até $10^6$.
\hfill

\begin{multicols}{2}
\begin{lstlisting}[language=C++]
typedef complex<double> cd;
typedef vector<cd> poly;
const double PI = acos(-1);

void fft(poly &a, bool invert = 0) {
    int n = a.size(), log_n = 0;
    while ((1 << log_n) < n) {
        log_n++;
    }

    for (int i = 1, j = 0; i < n; ++i) {
        int bit = n >> 1;
        for (; j >= bit; bit >>= 1) {
            j -= bit;
        }
        j += bit;
        if (i < j) {
            swap(a[i], a[j]);
        }
    }

    double angle = 2 * PI / n * (invert ? -1 : 1);
    poly root(n / 2);
    for (int i = 0; i < n / 2; ++i) {
        root[i] = cd(cos(angle * i), sin(angle * i));
    }

    for (long long len = 2; len <= n; len <<= 1) {
        long long step = n / len;
        long long aux = len / 2;
        for (long long i = 0; i < n; i += len) {
            for (int j = 0; j < aux; ++j) {
                cd u = a[i + j], v = a[i + j + aux] * root[step * j];
                a[i + j] = u + v;
                a[i + j + aux] = u - v;
            }
        }
    }
    if (invert) {
        for (int i = 0; i < n; ++i) {
            a[i] /= n;
        }
    }
}

vector<long long> convolution(vector<long long> &a, vector<long long> &b) {
    int n = 1, len = a.size() + b.size();
    while (n < len) {
        n <<= 1;
    }
    a.resize(n);
    b.resize(n);
    poly fft_a(a.begin(), a.end());
    fft(fft_a);
    poly fft_b(b.begin(), b.end());
    fft(fft_b);

    poly c(n);
    for (int i = 0; i < n; ++i) {
        c[i] = fft_a[i] * fft_b[i];
    }
    fft(c, 1);

    vector<long long> res(n);
    for (int i = 0; i < n; ++i) {
        res[i] = round(c[i].real()); //  res = c[i].real();
                                     //  se for vector de
                                     //  double
    }
    // while(size(res) > 1 && res.back() == 0)
    // res.pop_back(); // apenas para quando os
    // zeros direita nao importarem
    return res;
}
\end{lstlisting}
\end{multicols}

\hfill

\section{Primos}


Algortimos relacionados a números primos.



\textbf{Crivo de Eratóstenes} 

Computa a primalidade de todos os números até N, quase tão rápido quanto o crivo linear.



\begin{itemize}
\item Complexidade de tempo: O(N * log(log(N)))
\end{itemize}



Demora 1 segundo para LIM igual a $3 * 10^7$.



\textbf{Miller-Rabin} 

Teste de primalidade garantido para números menores do que $2^64$.



\begin{itemize}
\item Complexidade de tempo: O(log(N))
\end{itemize}



\textbf{Teste Ingênuo} 

Computa a primalidade de um número N.



\begin{itemize}
\item Complexidade de tempo: $O(N^(1/2))$
\end{itemize}

\hfill

\begin{multicols}{2}
\begin{lstlisting}[language=C++]
long long power(long long base, long long e, long long mod) {
    long long result = 1;
    base %= mod;
    while (e) {
        if (e & 1) {
            result = (__int128)result * base % mod;
        }
        base = (__int128)base * base % mod;
        e >>= 1;
    }
    return result;
}

bool is_composite(long long n, long long a, long long d, int s) {
    long long x = power(a, d, n);
    if (x == 1 || x == n - 1) {
        return false;
    }
    for (int r = 1; r < s; r++) {
        x = (__int128)x * x % n;
        if (x == n - 1) {
            return false;
        }
    }
    return true;
}

bool miller_rabin(long long n) {
    if (n < 2) {
        return false;
    }
    int r = 0;
    long long d = n - 1;
    while ((d & 1) == 0) {
        d >>= 1, ++r;
    }
    for (int a : {2, 3, 5, 7, 11, 13, 17, 19, 23, 29, 31, 37}) {
        if (n == a) {
            return true;
        }
        if (is_composite(n, a, d, r)) {
            return false;
        }
    }
    return true;
}
\end{lstlisting}
\end{multicols}

\hfill

\begin{multicols}{2}
\begin{lstlisting}[language=C++]
bool is_prime(int n) {
    for (long long d = 2; d * d <= n; d++) {
        if (n % d == 0) {
            return false;
        }
    }
    return true;
}
\end{lstlisting}
\end{multicols}

\hfill

\begin{multicols}{2}
\begin{lstlisting}[language=C++]
vector<bool> sieve(int n) {
    vector<bool> is_prime(n + 5, true);
    is_prime[0] = false;
    is_prime[1] = false;
    long long sq = sqrt(n + 5);
    for (long long i = 2; i <= sq; i++) {
        if (is_prime[i]) {
            for (long long j = i * i; j < n; j += i) {
                is_prime[j] = false;
            }
        }
    }
    return is_prime;
}
\end{lstlisting}
\end{multicols}

\hfill

\section{Inverso Modular}


Algoritmos para calcular o inverso modular de um número. O inverso modular de um inteiro $a$ é outro inteiro $x$ tal que $a \cdot x \equiv 1 \pmod{MOD}$



The modular inverse of an integer \lstinline{a} is another integer \lstinline{x} such that \lstinline{a * x} is congruent to \lstinline{1 (mod MOD)}.



\textbf{Modular Inverse} 



Calculates the modular inverse of \lstinline{a}.



Uses the [exp\_mod](/Matemática/Exponenciação\%20Modular\%20Rápida/exp\_mod.cpp) algorithm, thus expects \lstinline{MOD} to be prime.



* Time Complexity: O(log(MOD)).

* Space Complexity: O(1).



\textbf{Modular Inverse by Extended GDC} 



Calculates the modular inverse of \lstinline{a}.



Uses the [extended\_gcd](/Matemática/GCD/extended\_gcd.cpp) algorithm, thus expects \lstinline{MOD} to be coprime with \lstinline{a}.



Returns \lstinline{-1} if this assumption is broken.



* Time Complexity: O(log(MOD)).

* Space Complexity: O(1).



\textbf{Modular Inverse for 1 to MAX} 



Calculates the modular inverse for all numbers between \lstinline{1} and \lstinline{MAX}.



expects \lstinline{MOD} to be prime.



* Time Complexity: O(MAX).

* Space Complexity: O(MAX).



\textbf{Modular Inverse for all powers} 



Let \lstinline{b} be any integer.



Calculates the modular inverse for all powers of \lstinline{b} between \lstinline{b^0} and \lstinline{b^MAX}.



Needs you calculate beforehand the modular inverse of \lstinline{b}, for 2 it is always \lstinline{(MOD+1)/2}.



expects \lstinline{MOD} to be coprime with \lstinline{b}.



* Time Complexity: O(MAX).

* Space Complexity: O(MAX).

\hfill

\begin{multicols}{2}
\begin{lstlisting}[language=C++]
int inv(int a) {
    int x, y;
    int g = extended_gcd(a, MOD, x, y);
    if (g == 1) {
        return (x % m + m) % m;
    }
    return -1;
}
\end{lstlisting}
\end{multicols}

\hfill

\begin{multicols}{2}
\begin{lstlisting}[language=C++]
ll inv[MAX];

void compute_inv(const ll m = MOD) {
    inv[1] = 1;
    for (int i = 2; i < MAX; i++) {
        inv[i] = m - (m / i) * inv[m % i] % m;
    }
}
\end{lstlisting}
\end{multicols}

\hfill

\begin{multicols}{2}
\begin{lstlisting}[language=C++]
ll inv(ll a) {
    return exp_mod(a, MOD - 2);
}
\end{lstlisting}
\end{multicols}

\hfill

\begin{multicols}{2}
\begin{lstlisting}[language=C++]
const ll INVB = (MOD + 1) / 2; // Modular inverse of the base,
                               // for 2 it is (MOD+1)/2

ll inv[MAX]; // Modular inverse of b^i

void compute_inv() {
    inv[0] = 1;
    for (int i = 1; i < MAX; i++) {
        inv[i] = inv[i - 1] * INVB % MOD;
    }
}
\end{lstlisting}
\end{multicols}

\hfill

\section{NTT}


Computa a multiplicação de polinômios com coeficientes inteiros módulo um número primo.



Computa multiplicação de polinômino; \textbf{Somente para inteiros}.



\begin{itemize}
\item Complexidade de tempo: O(N * log(N))
\end{itemize}



Constantes finais devem ser menor do que $10^9$.  

Para constantes entre $10^9$ e $10^{18}$ é necessário codar também [big\_convolution](big\_convolution.cpp).

\hfill

\begin{multicols}{2}
\begin{lstlisting}[language=C++]
typedef long long ll;
typedef vector<ll> poly;

ll mod[3] = {998244353LL, 1004535809LL, 1092616193LL};
ll root[3] = {102292LL, 12289LL, 23747LL};
ll root_1[3] = {116744195LL, 313564925LL, 642907570LL};
ll root_pw[3] = {1LL << 23, 1LL << 21, 1LL << 21};

ll modInv(ll b, ll m) {
    ll e = m - 2;
    ll res = 1;
    while (e) {
        if (e & 1) {
            res = (res * b) % m;
        }
        e /= 2;
        b = (b * b) % m;
    }
    return res;
}

void ntt(poly &a, bool invert, int id) {
    ll n = (ll)a.size(), m = mod[id];
    for (ll i = 1, j = 0; i < n; ++i) {
        ll bit = n >> 1;
        for (; j >= bit; bit >>= 1) {
            j -= bit;
        }
        j += bit;
        if (i < j) {
            swap(a[i], a[j]);
        }
    }
    for (ll len = 2, wlen; len <= n; len <<= 1) {
        wlen = invert ? root_1[id] : root[id];
        for (ll i = len; i < root_pw[id]; i <<= 1) {
            wlen = (wlen * wlen) % m;
        }
        for (ll i = 0; i < n; i += len) {
            ll w = 1;
            for (ll j = 0; j < len / 2; j++) {
                ll u = a[i + j], v = (a[i + j + len / 2] * w) % m;
                a[i + j] = (u + v) % m;
                a[i + j + len / 2] = (u - v + m) % m;
                w = (w * wlen) % m;
            }
        }
    }
    if (invert) {
        ll inv = modInv(n, m);
        for (ll i = 0; i < n; i++) {
            a[i] = (a[i] * inv) % m;
        }
    }
}

poly convolution(poly a, poly b, int id = 0) {
    ll n = 1LL, len = (1LL + a.size() + b.size());
    while (n < len) {
        n *= 2;
    }
    a.resize(n);
    b.resize(n);
    ntt(a, 0, id);
    ntt(b, 0, id);
    poly answer(n);
    for (ll i = 0; i < n; i++) {
        answer[i] = (a[i] * b[i]);
    }
    ntt(answer, 1, id);
    return answer;
}
\end{lstlisting}
\end{multicols}

\hfill

\begin{multicols}{2}
\begin{lstlisting}[language=C++]

ll mod_mul(ll a, ll b, ll m) {
    return (__int128)a * b % m;
}
ll ext_gcd(ll a, ll b, ll &x, ll &y) {
    if (!b) {
        x = 1;
        y = 0;
        return a;
    } else {
        ll g = ext_gcd(b, a % b, y, x);
        y -= a / b * x;
        return g;
    }
}

// convolution mod 1,097,572,091,361,755,137
poly big_convolution(poly a, poly b) {
    poly r0, r1, answer;
    r0 = convolution(a, b, 1);
    r1 = convolution(a, b, 2);

    ll s, r, p = mod[1] * mod[2];
    ext_gcd(mod[1], mod[2], r, s);

    answer.resize(r0.size());
    for (int i = 0; i < (int)answer.size(); i++) {
        answer[i] = (mod_mul((s * mod[2] + p) % p, r0[i], p) +
                     mod_mul((r * mod[1] + p) % p, r1[i], p) + p) %
                    p;
    }
    return answer;
}
\end{lstlisting}
\end{multicols}

\hfill

\section{Teorema do Resto Chinês}


Algoritmo que resolve o sistema $x \equiv a_i \pmod{m_i}$, onde $m_i$ são primos entre si.



\textbf{} 
\hfill

\begin{multicols}{2}
\begin{lstlisting}[language=C++]
ll extended_gcd(ll a, ll b, ll &x, ll &y) {
    if (b == 0) {
        x = 1;
        y = 0;
        return a;
    } else {
        ll g = extended_gcd(b, a % b, y, x);
        y -= a / b * x;
        return g;
    }
}

ll crt(vector<ll> rem, vector<ll> mod) {
    int n = rem.size();
    if (n == 0) {
        return 0;
    }
    __int128 ans = rem[0], m = mod[0];
    for (int i = 1; i < n; i++) {
        ll x, y;
        ll g = extended_gcd(mod[i], m, x, y);
        if ((ans - rem[i]) % g != 0) {
            return -1;
        }
        ans = ans + (__int128)1 * (rem[i] - ans) * (m / g) * y;
        m = (__int128)(mod[i] / g) * (m / g) * g;
        ans = (ans % m + m) % m;
    }
    return ans;
}
\end{lstlisting}
\end{multicols}

\hfill

\section{Fatoração}


Algortimos para fatorar um número.



\textbf{Fatoração Simples} 

Fatora um número N.



\begin{itemize}
\item Complexidade de tempo: $O(\sqrt n )$
\end{itemize}



\textbf{Crivo Linear} 

Pré-computa todos os fatores primos até MAX.

Utilizado para fatorar um número N menor que MAX.



\begin{itemize}
\item Complexidade de tempo: Pré-processamento O(MAX)
\item Complexidade de tempo: Fatoraração O(quantidade de fatores de N)
\item Complexidade de espaço: O(MAX)
\end{itemize}



\textbf{Fatoração Rápida} 

Utiliza Pollar-Rho e Miller-Rabin (ver em Primos) para fatorar um número N.



\begin{itemize}
\item Complexidade de tempo: $O(N^{1/4} \cdot log(N))$
\end{itemize}



\textbf{Pollard-Rho} 

Descobre um divisor de um número N.



\begin{itemize}
\item Complexidade de tempo: $O(N^{1/4} \cdot log(N))$
\item Complexidade de espaço: $O(N^{1/2})$
\end{itemize}

\hfill

\begin{multicols}{2}
\begin{lstlisting}[language=C++]
// usa miller_rabin.cpp!! olhar em
// matematica/primos usa pollar_rho.cpp!! olhar em
// matematica/fatoracao

vector<long long> factorize(long long n) {
    if (n == 1) {
        return {};
    }
    if (miller_rabin(n)) {
        return {n};
    }
    long long x = pollard_rho(n);
    auto l = factorize(x), r = factorize(n / x);
    l.insert(l.end(), all(r));
    return l;
}
\end{lstlisting}
\end{multicols}

\hfill

\begin{multicols}{2}
\begin{lstlisting}[language=C++]
long long mod_mul(long long a, long long b, long long m) {
    return (__int128)a * b % m;
}

long long pollard_rho(long long n) {
    auto f = [n](long long x) {
        return mod_mul(x, x, n) + 1;
    };
    long long x = 0, y = 0, t = 30, prd = 2, i = 1, q;
    while (t++ % 40 || __gcd(prd, n) == 1) {
        if (x == y) {
            x = ++i, y = f(x);
        }
        if ((q = mod_mul(prd, max(x, y) - min(x, y), n))) {
            prd = q;
        }
        x = f(x), y = f(f(y));
    }
    return __gcd(prd, n);
}
\end{lstlisting}
\end{multicols}

\hfill

\begin{multicols}{2}
\begin{lstlisting}[language=C++]
namespace sieve {
    const int MAX = 1e4;
    int lp[MAX + 1], factor[MAX + 1];
    vector<int> pr;
    void build() {
        for (int i = 2; i <= MAX; ++i) {
            if (lp[i] == 0) {
                lp[i] = i;
                pr.push_back(i);
            }
            for (int j = 0; i * pr[j] <= MAX; ++j) {
                lp[i * pr[j]] = pr[j];
                factor[i * pr[j]] = i;
                if (pr[j] == lp[i]) {
                    break;
                }
            }
        }
    }
    vector<int> factorize(int x) {
        if (x < 2) {
            return {};
        }
        vector<int> v;
        for (int lpx = lp[x]; x >= lpx; x = factor[x]) {
            v.emplace_back(lp[x]);
        }
        return v;
    }
}
\end{lstlisting}
\end{multicols}

\hfill

\begin{multicols}{2}
\begin{lstlisting}[language=C++]
vector<int> factorize(int n) {
    vector<int> factors;
    for (long long d = 2; d * d <= n; d++) {
        while (n % d == 0) {
            factors.push_back(d);
            n /= d;
        }
    }
    if (n != 1) {
        factors.push_back(n);
    }
    return factors;
}
\end{lstlisting}
\end{multicols}

\hfill

\section{Totiente de Euler}


Código para computar o totiente de Euler.



\textbf{Totiente de Euler (Phi) para um número} 

Computa o totiente para um único número N.



\begin{itemize}
\item Complexidade de tempo: $O(N^(1/2))$
\end{itemize}



\textbf{Totiente de Euler (Phi) entre 1 e N} 

Computa o totiente entre 1 e N.



\begin{itemize}
\item Complexidade de tempo: O(N * log(log(N)))
\end{itemize}

\hfill

\begin{multicols}{2}
\begin{lstlisting}[language=C++]
vector<int> phi_1_to_n(int n) {
    vector<int> phi(n + 1);
    for (int i = 0; i <= n; i++) {
        phi[i] = i;
    }
    for (int i = 2; i <= n; i++) {
        if (phi[i] == i) {
            for (int j = i; j <= n; j += i) {
                phi[j] -= phi[j] / i;
            }
        }
    }
    return phi;
}
\end{lstlisting}
\end{multicols}

\hfill

\begin{multicols}{2}
\begin{lstlisting}[language=C++]
int phi(int n) {
    int result = n;
    for (int i = 2; i * i <= n; i++) {
        if (n % i == 0) {
            while (n % i == 0) {
                n /= i;
            }
            result -= result / i;
        }
    }
    if (n > 1) {
        result -= result / n;
    }
    return result;
}
\end{lstlisting}
\end{multicols}

\hfill

\section{GCD}


Algoritmo Euclides para computar o Máximo Divisor Comum (MDC em português; GCD em inglês), e variações.



*Read in [English](README.en.md)*



\textbf{Algoritmo de Euclides} 



Computa o Máximo Divisor Comum (MDC em português; GCD em inglês).



\begin{itemize}
\item Complexidade de tempo: O(log(n))
\end{itemize}



Mais demorado que usar a função do compilador C++ \_\_gcd(a,b).



\textbf{Algoritmo de Euclides Estendido} 



Algoritmo extendido de euclides que computa o Máximo Divisor Comum e os valores x e y tal que a * x + b * y = gcd(a, b).



\begin{itemize}
\item Complexidade de tempo: O(log(n))
\end{itemize}

\hfill

\begin{multicols}{2}
\begin{lstlisting}[language=C++]
ll extended_gcd(ll a, ll b, ll &x, ll &y) {
    if (b == 0) {
        x = 1;
        y = 0;
        return a;
    } else {
        ll g = extended_gcd(b, a % b, y, x);
        y -= a / b * x;
        return g;
    }
}
\end{lstlisting}
\end{multicols}

\hfill

\begin{multicols}{2}
\begin{lstlisting}[language=C++]
int extended_gcd(int a, int b, int &x, int &y) {
    x = 1, y = 0;
    int x1 = 0, y1 = 1;
    while (b) {
        int q = a / b;
        tie(x, x1) = make_tuple(x1, x - q * x1);
        tie(y, y1) = make_tuple(y1, y - q * y1);
        tie(a, b) = make_tuple(b, a - q * b);
    }
    return a;
}
\end{lstlisting}
\end{multicols}

\hfill

\begin{multicols}{2}
\begin{lstlisting}[language=C++]
long long gcd(long long a, long long b) {
    return (b == 0) ? a : gcd(b, a % b);
}
\end{lstlisting}
\end{multicols}

\hfill

\section{Exponenciação Modular Rápida}


Computa $(base^{exp}) \% mod$.



\begin{itemize}
\item Complexidade de tempo: $O(log(exp))$.
\item Complexidade de espaço: $O(1)$
\end{itemize}

\hfill

\begin{multicols}{2}
\begin{lstlisting}[language=C++]
ll exp_mod(ll base, ll exp) {
    ll b = base, res = 1;
    while (exp) {
        if (exp & 1) {
            res = (res * b) % MOD;
        }
        b = (b * b) % MOD;
        exp /= 2;
    }
    return res;
}
\end{lstlisting}
\end{multicols}

\hfill

\newpage

%%%%%%
%
%
% Estruturas de Dados
%
%
%%%%%%

\chapter{Estruturas de Dados}

\section{Fenwick Tree}


Consultas e atualizações de soma em intervalo.



O vetor precisa obrigatoriamente estar indexado em 1.



* Complexidade de tempo (Pre-processamento): $O(N * log(N))$

* Complexidade de tempo (Consulta em intervalo): $O(log(N))$

* Complexidade de tempo (Update em ponto): $O(log(N))$

* Complexidade de espaço: 2 * N = $O(N)$
\hfill

\begin{multicols}{2}
\begin{lstlisting}[language=C++]
struct FenwickTree {
    int n;
    vector<int> tree;
    FenwickTree(int n) : n(n) {
        tree.assign(n, 0);
    }
    FenwickTree(vector<int> v) : FenwickTree(v.size()) {
        for (size_t i = 1; i < v.size(); i++) {
            update(i, v[i]);
        }
    }
    int lsONE(int x) {
        return x & (-x);
    }
    int query(int x) {
        int soma = 0;
        for (; x > 0; x -= lsONE(x)) {
            soma += tree[x];
        }
        return soma;
    }
    int query(int l, int r) {
        return query(r) - query(l - 1);
    }
    void update(int x, int v) {
        for (; x < n; x += lsONE(x)) {
            tree[x] += v;
        }
    }
};
\end{lstlisting}
\end{multicols}

\hfill

\section{MergeSort Tree}


Árvore que resolve queries que envolvam ordenação em range.



\begin{itemize}
\item Complexidade de construção : $O(N * log(N))$
\item Complexidade de consulta : $O(log^2(N))$
\end{itemize}



\textbf{MergeSort Tree com Update Pontual} 



Resolve Queries que envolvam ordenação em Range. (\textbf{COM UPDATE})  

\textbf{1 segundo para vetores de tamanho $3 * 10^5$}



\begin{itemize}
\item Complexidade de construção : $O(N * log^2(N))$
\item Complexidade de consulta : $O(log^2(N))$
\item Complexidade de update : $O(log^2(N))$
\end{itemize}

\hfill

\begin{multicols}{2}
\begin{lstlisting}[language=C++]
namespace mergesort {
    const int MAX = 1e5 + 5;

    int n;
    vi mgtree[4 * MAX];

    int le(int n) {
        return 2 * n + 1;
    }
    int ri(int n) {
        return 2 * n + 2;
    }

    void build(int n, int esq, int dir, vi &v) {
        mgtree[n] = vi(dir - esq + 1, 0);
        if (esq == dir) {
            mgtree[n][0] = v[esq];
        } else {
            int mid = (esq + dir) / 2;
            build(le(n), esq, mid, v);
            build(ri(n), mid + 1, dir, v);
            merge(mgtree[le(n)].begin(),
                  mgtree[le(n)].end(),
                  mgtree[ri(n)].begin(),
                  mgtree[ri(n)].end(),
                  mgtree[n].begin());
        }
    }
    void build(vi &v) {
        n = v.size();
        build(0, 0, n - 1, v);
    }

    int less(int n, int esq, int dir, int l, int r, int k) {
        if (esq > r || dir < l) {
            return 0;
        }
        if (l <= esq && dir <= r) {
            return lower_bound(mgtree[n].begin(), mgtree[n].end(), k) - mgtree[n].begin();
        }
        int mid = (esq + dir) / 2;
        return less(le(n), esq, mid, l, r, k) + less(ri(n), mid + 1, dir, l, r, k);
    }
    int less(int l, int r, int k) {
        return less(0, 0, n - 1, l, r, k);
    }

    // vi debug_query(int n, int esq, int dir, int
    // l, int r) {
    //     if (esq > r || dir < l) return vi();
    //     if (l <= esq && dir <= r) return
    //     mgtree[n]; int mid = (esq + dir) / 2;
    //     auto vl = debug_query(le(n), esq, mid,
    //     l, r); auto vr = debug_query(ri(n),
    //     mid+1, dir, l, r); vi ans =
    //     vi(vl.size() + vr.size());
    //     merge(vl.begin(), vl.end(),
    //         vr.begin(), vr.end(),
    //         ans.begin());
    //     return ans;
    // }
    // vi debug_query(int l, int r) {return
    // debug_query(0, 0, n-1, l, r);}
};
\end{lstlisting}
\end{multicols}

\hfill

\begin{multicols}{2}
\begin{lstlisting}[language=C++]
#include <ext/pb_ds/assoc_container.hpp>
#include <ext/pb_ds/tree_policy.hpp>

using namespace __gnu_pbds;

namespace mergesort {
    typedef tree<ii, null_type, less<ii>, rb_tree_tag, tree_order_statistics_node_update>
        ordered_set;
    const int MAX = 1e5 + 5;

    int n;
    ordered_set mgtree[4 * MAX];
    vi values;

    int le(int n) {
        return 2 * n + 1;
    }
    int ri(int n) {
        return 2 * n + 2;
    }

    ordered_set join(ordered_set set_l, ordered_set set_r) {
        for (auto v : set_r) {
            set_l.insert(v);
        }
        return set_l;
    }

    void build(int n, int esq, int dir) {
        if (esq == dir) {
            mgtree[n].insert(ii(values[esq], esq));
        } else {
            int mid = (esq + dir) / 2;
            build(le(n), esq, mid);
            build(ri(n), mid + 1, dir);
            mgtree[n] = join(mgtree[le(n)], mgtree[ri(n)]);
        }
    }
    void build(vi &v) {
        n = v.size();
        values = v;
        build(0, 0, n - 1);
    }

    int less(int n, int esq, int dir, int l, int r, int k) {
        if (esq > r || dir < l) {
            return 0;
        }
        if (l <= esq && dir <= r) {
            return mgtree[n].order_of_key({k, -1});
        }
        int mid = (esq + dir) / 2;
        return less(le(n), esq, mid, l, r, k) + less(ri(n), mid + 1, dir, l, r, k);
    }
    int less(int l, int r, int k) {
        return less(0, 0, n - 1, l, r, k);
    }

    void update(int n, int esq, int dir, int x, int v) {
        if (esq > x || dir < x) {
            return;
        }
        if (esq == dir) {
            mgtree[n].clear(), mgtree[n].insert(ii(v, x));
        } else {
            int mid = (esq + dir) / 2;
            if (x <= mid) {
                update(le(n), esq, mid, x, v);
            } else {
                update(ri(n), mid + 1, dir, x, v);
            }
            mgtree[n].erase(ii(values[x], x));
            mgtree[n].insert(ii(v, x));
        }
    }
    void update(int x, int v) {
        update(0, 0, n - 1, x, v);
        values[x] = v;
    }

    // ordered_set debug_query(int n, int esq, int
    // dir, int l, int r) {
    //     if (esq > r || dir < l) return
    //     ordered_set(); if (l <= esq && dir <=
    //     r) return mgtree[n]; int mid = (esq +
    //     dir) / 2; return
    //     join(debug_query(le(n), esq, mid, l,
    //     r), debug_query(ri(n), mid+1, dir, l,
    //     r));
    // }
    // ordered_set debug_query(int l, int r)
    // {return debug_query(0, 0, n-1, l, r);}

    // int greater(int n, int esq, int dir, int l,
    // int r, int k) {
    //     if (esq > r || dir < l) return 0;
    //     if (l <= esq && dir <= r) return
    //     (r-l+1) - mgtree[n].order_of_key({k,
    //     1e8}); int mid = (esq + dir) / 2;
    //     return greater(le(n), esq, mid, l, r,
    //     k) + greater(ri(n), mid+1, dir, l, r,
    //     k);
    // }
    // int greater(int l, int r, int k) {return
    // greater(0, 0, n-1, l, r, k);}
};
\end{lstlisting}
\end{multicols}

\hfill

\section{Operation Queue}


Fila que armazena o resultado do operatório dos itens.



* Complexidade de tempo (Push): O(1)

* Complexidade de tempo (Pop): O(1)

\hfill

\begin{multicols}{2}
\begin{lstlisting}[language=C++]
template <typename T> struct op_queue {
    stack<pair<T, T>> s1, s2;
    T result;
    T op(T a, T b) {
        return a; // TODO: op to compare
        // min(a, b);
        // gcd(a, b);
        // lca(a, b);
    }
    T get() {
        if (s1.empty() || s2.empty()) {
            return result = s1.empty() ? s2.top().second : s1.top().second;
        } else {
            return result = op(s1.top().second, s2.top().second);
        }
    }
    void add(T element) {
        result = s1.empty() ? element : op(element, s1.top().second);
        s1.push({element, result});
    }
    void remove() {
        if (s2.empty()) {
            while (!s1.empty()) {
                T elem = s1.top().first;
                s1.pop();
                T result = s2.empty() ? elem : op(elem, s2.top().second);
                s2.push({elem, result});
            }
        }
        T remove_elem = s2.top().first;
        s2.pop();
    }
};
\end{lstlisting}
\end{multicols}

\hfill

\section{Ordered Set}


Set com operações de busca por ordem e índice.



Pode ser usado como um set normal, a principal diferença são duas novas operações possíveis:



\begin{itemize}
\item \lstinline{find_by_order(x)}: retorna o item na posição x.
\item \lstinline{order_of_key(k)}: retorna o número de elementos menores que k. (o índice de k)
\end{itemize}



\textbf{} 
\begin{lstlisting}[language=C++]
#include <ext/pb_ds/assoc_container.hpp>
#include <ext/pb_ds/trie_policy.hpp>

using namespace __gnu_pbds;
typedef tree<int, null_type, less<int>, rb_tree_tag, tree_order_statistics_node_update> ordered_set;

ordered_set X;
X.insert(1);
X.insert(2);
X.insert(4);
X.insert(8);
X.insert(16);

cout<<*X.find_by_order(1)<<endl; // 2
cout<<*X.find_by_order(2)<<endl; // 4
cout<<*X.find_by_order(4)<<endl; // 16
cout<<(end(X)==X.find_by_order(6))<<endl; // true

cout<<X.order_of_key(-5)<<endl;  // 0
cout<<X.order_of_key(1)<<endl;   // 0
cout<<X.order_of_key(3)<<endl;   // 2
cout<<X.order_of_key(4)<<endl;   // 2
cout<<X.order_of_key(400)<<endl; // 5

\end{lstlisting}

\hfill

\begin{multicols}{2}
\begin{lstlisting}[language=C++]
#include <ext/pb_ds/assoc_container.hpp>
#include <ext/pb_ds/trie_policy.hpp>

using namespace __gnu_pbds;

template <typename T>
typedef tree<T, null_type, less<T>, rb_tree_tag, tree_order_statistics_node_update>
    ordered_set;\end{lstlisting}
\end{multicols}

\hfill

\section{Disjoint Set Union}
\subsection{DSU Rollback}


Desfaz as últimas K uniões



\begin{itemize}
\item Complexidade de tempo: O(K).
\end{itemize}



É possivel usar um checkpoint, bastando chamar \textbf{rollback()} para ir até o último checkpoint.  

O rollback não altera a complexidade, uma vez que K <= queries.  

\textbf{Só funciona sem compressão de caminho}



\begin{itemize}
\item Complexidade de tempo: O(log(N))
\end{itemize}

\hfill

\begin{multicols}{2}
\begin{lstlisting}[language=C++]
struct rollback_dsu {
    struct change {
        int node, old_size;
    };
    stack<change> changes;
    vector<int> parent, size;
    int number_of_sets;

    rollback_dsu(int n) {
        size.resize(n + 5, 1);
        number_of_sets = n;
        for (int i = 0; i < n + 5; ++i) {
            parent.push_back(i);
        }
    }

    int get(int a) {
        return (a == parent[a]) ? a : get(parent[a]);
    }
    bool same(int a, int b) {
        return get(a) == get(b);
    }
    void checkpoint() {
        changes.push({-2, 0});
    }

    void join(int a, int b) {
        a = get(a);
        b = get(b);
        if (a == b) {
            changes.push({-1, -1});
            return;
        }
        if (size[a] > size[b]) {
            swap(a, b);
        }
        changes.push({a, size[b]});
        parent[a] = b;
        size[b] += size[a];
        --number_of_sets;
    }

    void rollback(int qnt = 1 << 31) {
        for (int i = 0; i < qnt; ++i) {
            auto ch = changes.top();
            changes.pop();
            if (ch.node == -1) {
                continue;
            }
            if (ch.node == -2) {
                if (qnt == 1 << 31) {
                    break;
                }
                --i;
                continue;
            }
            size[parent[ch.node]] = ch.old_size;
            parent[ch.node] = ch.node;
            ++number_of_sets;
        }
    }
};
\end{lstlisting}
\end{multicols}

\hfill

\subsection{DSU Bipartido}


DSU para grafo bipartido, é possível verificar se uma aresta é possível antes de adicioná-la.  

Para todas as operações:



\begin{itemize}
\item Complexidade de tempo: O(1) amortizado.
\end{itemize}

\hfill

\begin{multicols}{2}
\begin{lstlisting}[language=C++]
struct bipartite_dsu {
    vector<int> parent;
    vector<int> color;
    int size;
    bipartite_dsu(int n) {
        size = n;
        color.resize(n + 5, 0);
        for (int i = 0; i < n + 5; ++i) {
            parent.push_back(i);
        }
    }

    pair<int, bool> get(int a) {
        if (parent[a] == a) {
            return {a, 0};
        }
        auto val = get(parent[a]);
        parent[a] = val.fi;
        color[a] = (color[a] + val.se) % 2;
        return {parent[a], color[a]};
    }

    bool same_color(int a, int b) {
        get(a);
        get(b);
        return color[a] == color[b];
    }
    bool same_group(int a, int b) {
        get(a);
        get(b);
        return parent[a] == parent[b];
    }
    bool possible_edge(int a, int b) {
        return !same_color(a, b) || !same_group(a, b);
    }

    void join(int a, int b) {
        auto val_a = get(a), val_b = get(b);
        parent[val_a.fi] = val_b.fi;
        color[val_a.fi] = (val_a.se + val_b.se + 1) % 2;
    }
};
\end{lstlisting}
\end{multicols}

\hfill

\subsection{DSU Simples}


Verifica se dois itens pertencem a um mesmo grupo.



\begin{itemize}
\item Complexidade de tempo: O(1) amortizado.
\end{itemize}



Une grupos.



\begin{itemize}
\item Complexidade de tempo: O(1) amortizado.
\end{itemize}

\hfill

\begin{multicols}{2}
\begin{lstlisting}[language=C++]
struct DSU {
    vector<int> pa, sz;
    DSU(int n) : pa(n + 1), sz(n + 1, 1) {
        iota(pa.begin(), pa.end(), 0);
    }
    int root(int a) {
        return pa[a] = (a == pa[a] ? a : root(pa[a]));
    }
    bool find(int a, int b) {
        return root(a) == root(b);
    }
    void uni(int a, int b) {
        int ra = root(a), rb = root(b);
        if (ra == rb) {
            return;
        }
        if (sz[ra] > sz[rb]) {
            swap(ra, rb);
        }
        pa[ra] = rb;
        sz[rb] += sz[ra];
    }
};
\end{lstlisting}
\end{multicols}

\hfill

\subsection{DSU Completo}


DSU com capacidade de adicionar e remover vértices.  



\textbf{EXTREMAMENTE PODEROSO!}  



Funciona de maneira off-line, recebendo as operações e dando as respostas das consultas no retorno da função \textbf{solve()}



\begin{itemize}
\item Complexidade de tempo: O(Q * log(Q) * log(N)); Onde Q é o número de consultas e N o número de nodos
\end{itemize}



Roda em 0,6ms para $3 * 10^5$ queries e nodos com printf e scanf.  

Possivelmente aguenta $10^6$ em 3s
\hfill

\begin{multicols}{2}
\begin{lstlisting}[language=C++]
struct full_dsu {
    struct change {
        int node, old_size;
    };
    struct query {
        int l, r, u, v, type;
    };
    stack<change> changes;
    map<pair<int, int>, vector<query>> edges;
    vector<query> queries;
    vector<int> parent, size;
    int number_of_sets, time;

    full_dsu(int n) {
        time = 0;
        size.resize(n + 5, 1);
        number_of_sets = n;
        loop(i, 0, n + 5) parent.push_back(i);
    }

    int get(int a) {
        return (parent[a] == a ? a : get(parent[a]));
    }
    bool same(int a, int b) {
        return get(a) == get(b);
    }
    void checkpoint() {
        changes.push({-2, 0});
    }

    void join(int a, int b) {
        a = get(a);
        b = get(b);
        if (a == b) {
            return;
        }
        if (size[a] > size[b]) {
            swap(a, b);
        }
        changes.push({a, size[b]});
        parent[a] = b;
        size[b] += size[a];
        --number_of_sets;
    }

    void rollback() {
        while (!changes.empty()) {
            auto ch = changes.top();
            changes.pop();
            if (ch.node == -2) {
                break;
            }
            size[parent[ch.node]] = ch.old_size;
            parent[ch.node] = ch.node;
            ++number_of_sets;
        }
    }

    void ord(int &a, int &b) {
        if (a > b) {
            swap(a, b);
        }
    }

    void add(int u, int v) {
        ord(u, v);
        edges[{u, v}].push_back({time++, (int)1e9, u, v, 0});
    }

    void remove(int u, int v) {
        ord(u, v);
        edges[{u, v}].back().r = time++;
    }

    // consulta se dois vertices estao no mesmo
    // grupo
    void question(int u, int v) {
        ord(u, v);
        queries.push_back({time, time, u, v, 1});
        ++time;
    }

    // consulta a quantidade de grupos distintos
    void question() {
        queries.push_back({time, time, 0, 0, 1});
        ++time;
    }

    vector<int> solve() {
        for (auto [p, v] : edges) {
            queries.insert(queries.end(), all(v));
        }
        vector<int> vec(time, -1), ans;
        run(queries, 0, time, vec);
        for (int i : vec) {
            if (i != -1) {
                ans.push_back(i);
            }
        }
        return ans;
    }

    void run(const vector<query> &qrs, int l, int r, vector<int> &ans) {
        if (l > r) {
            return;
        }
        checkpoint();
        vector<query> qrs_aux;
        for (auto &q : qrs) {
            if (!q.type && q.l <= l && r <= q.r) {
                join(q.u, q.v);
            } else if (r < q.l || l > q.r) {
                continue;
            } else {
                qrs_aux.push_back(q);
            }
        }
        if (l == r) {
            for (auto &q : qrs) {
                if (q.type && q.l == l) {
                    ans[l] = number_of_sets;
                    // numero de grupos nesse tempo
                    // ans[l] = same(q.u, q.v);
                    // se u e v estao no mesmo grupo
                }
            }
            rollback();
            return;
        }
        int m = (l + r) / 2;
        run(qrs_aux, l, m, ans);
        run(qrs_aux, m + 1, r, ans);
        rollback();
    }
};
\end{lstlisting}
\end{multicols}

\hfill

\section{LiChao Tree}


Uma árvore de Funções. Retorna o F(x) máximo em um ponto X.



Para retornar o minimo deve-se inserir o negativo da função e pegar o negativo do resultado.



Está pronta para usar função linear do tipo F(x) = mx + b.



Funciona para funções com a seguinte propriedade, sejam duas funções f(x) e g(x), uma vez que f(x) ganha/perde de g(x), f(x) vai continuar ganhando/perdendo de g(x),

ou seja f(x) e g(x) se intersectam apenas uma vez.



* Complexidade de consulta : O(log(N))

* Complexidade de update: O(log(N))



\textbf{LiChao Tree Sparse} 



O mesmo que a superior, no entanto suporta consultas com |x| <= 1e18.



* Complexidade de consulta : O(log(tamanho do intervalo))

* Complexidade de update: O(log(tamanho do intervalo))

\hfill

\begin{multicols}{2}
\begin{lstlisting}[language=C++]
typedef long long ll;

const ll MAXN = 1e5 + 5, INF = 1e18 + 9;

struct Line {
    ll a, b = -INF;
    ll operator()(ll x) {
        return a * x + b;
    }
} tree[4 * MAXN];

int le(int n) {
    return 2 * n + 1;
}
int ri(int n) {
    return 2 * n + 2;
}

void insert(Line line, int n = 0, int l = 0, int r = MAXN) {
    int mid = (l + r) / 2;
    bool bl = line(l) < tree[n](l);
    bool bm = line(mid) < tree[n](mid);
    if (!bm) {
        swap(tree[n], line);
    }
    if (l == r) {
        return;
    }
    if (bl != bm) {
        insert(line, le(n), l, mid);
    } else {
        insert(line, ri(n), mid + 1, r);
    }
}

ll query(int x, int n = 0, int l = 0, int r = MAXN) {
    if (l == r) {
        return tree[n](x);
    }
    int mid = (l + r) / 2;
    if (x < mid) {
        return max(tree[n](x), query(x, le(n), l, mid));
    } else {
        return max(tree[n](x), query(x, ri(n), mid + 1, r));
    }
}
\end{lstlisting}
\end{multicols}

\hfill

\begin{multicols}{2}
\begin{lstlisting}[language=C++]
typedef long long ll;

const ll MAXN = 1e5 + 5, INF = 1e18 + 9, MAXR = 1e18;

struct Line {
    ll a, b = -INF;
    __int128 operator()(ll x) {
        return (__int128)a * x + b;
    }
} tree[4 * MAXN];
int idx = 0, L[4 * MAXN], R[4 * MAXN];

int le(int n) {
    if (!L[n]) {
        L[n] = ++idx;
    }
    return L[n];
}
int ri(int n) {
    if (!R[n]) {
        R[n] = ++idx;
    }
    return R[n];
}

void insert(Line line, int n = 0, ll l = -MAXR, ll r = MAXR) {
    ll mid = (l + r) / 2;
    bool bl = line(l) < tree[n](l);
    bool bm = line(mid) < tree[n](mid);
    if (!bm) {
        swap(tree[n], line);
    }
    if (l == r) {
        return;
    }
    if (bl != bm) {
        insert(line, le(n), l, mid);
    } else {
        insert(line, ri(n), mid + 1, r);
    }
}

__int128 query(int x, int n = 0, ll l = -MAXR, ll r = MAXR) {
    if (l == r) {
        return tree[n](x);
    }
    ll mid = (l + r) / 2;
    if (x < mid) {
        return max(tree[n](x), query(x, le(n), l, mid));
    } else {
        return max(tree[n](x), query(x, ri(n), mid + 1, r));
    }
}
\end{lstlisting}
\end{multicols}

\hfill

\section{Operation Stack}


Pilha que armazena o resultado do operatório dos itens.



* Complexidade de tempo (Push): O(1)

* Complexidade de tempo (Pop): O(1)

\hfill

\begin{multicols}{2}
\begin{lstlisting}[language=C++]
template <typename T> struct op_stack {
    stack<pair<T, T>> st;
    T result;
    T op(T a, T b) {
        return a; // TODO: op to compare
        // min(a, b);
        // gcd(a, b);
        // lca(a, b);
    }
    T get() {
        return result = st.top().second;
    }
    void add(T element) {
        result = st.empty() ? element : op(element, st.top().second);
        st.push({element, result});
    }
    void remove() {
        T removed_element = st.top().first;
        st.pop();
    }
};
\end{lstlisting}
\end{multicols}

\hfill

\section{Interval Tree}
\textbf{Por Rafael Granza de Mello}



Estrutura que trata intersecções de intervalos.



Capaz de retornar todos os intervalos que intersectam [L, R]. \textbf{L e R inclusos}  

Contém funções insert({L, R, ID}), erase({L, R, ID}) , overlaps(L, R) e find({L, R, ID}).  

É necessário inserir e apagar indicando tanto os limites quanto o ID do intervalo.



\begin{itemize}
\item Complexidade de tempo: O( N * log(N) ).
\end{itemize}



Podem ser usadas as operações em Set: 



\begin{itemize}
\item \lstinline{insert()}
\item \lstinline{erase()}
\item \lstinline{upper_bound()}
\item etc
\end{itemize}

\hfill

\begin{multicols}{2}
\begin{lstlisting}[language=C++]
#include <ext/pb_ds/assoc_container.hpp>
#include <ext/pb_ds/tree_policy.hpp>
using namespace __gnu_pbds;

struct interval {
    long long lo, hi, id;
    bool operator<(const interval &i) const {
        return lo < i.lo || (lo == i.lo && hi < i.hi) ||
               (lo == i.lo && hi == i.hi && id < i.id);
    }
};
template <class CNI, class NI, class Cmp_Fn, class Allocator>
struct intervals_node_update {
    typedef long long metadata_type;
    int sz = 0;
    virtual CNI node_begin() const = 0;
    virtual CNI node_end() const = 0;

    inline vector<int> overlaps(const long long l, const long long r) {
        queue<CNI> q;
        q.push(node_begin());
        vector<int> vec;
        while (!q.empty()) {
            CNI it = q.front();
            q.pop();
            if (it == node_end()) {
                continue;
            }
            if (r >= (*it)->lo && l <= (*it)->hi) {
                vec.push_back((*it)->id);
            }
            CNI l_it = it.get_l_child();
            long long l_max = (l_it == node_end()) ? -INF : l_it.get_metadata();
            if (l_max >= l) {
                q.push(l_it);
            }
            if ((*it)->lo <= r) {
                q.push(it.get_r_child());
            }
        }
        return vec;
    }

    inline void operator()(NI it, CNI end_it) {
        const long long l_max =
            (it.get_l_child() == end_it) ? -INF : it.get_l_child().get_metadata();
        const long long r_max =
            (it.get_r_child() == end_it) ? -INF : it.get_r_child().get_metadata();
        const_cast<long long &>(it.get_metadata()) = max((*it)->hi, max(l_max, r_max));
    }
};
typedef tree<interval, null_type, less<interval>, rb_tree_tag, intervals_node_update>
    interval_tree;
\end{lstlisting}
\end{multicols}

\hfill

\section{Sparse Table}
\subsection{Disjoint Sparse Table}


Resolve query de range para qualquer operação associativa em $O(1)$.



Pré-processamento em $O(n \log n)$.
\hfill

\begin{multicols}{2}
\begin{lstlisting}[language=C++]
struct dst {
    const int neutral = 1;
#define comp(a, b) (a | b)
    vector<vector<int>> t;
    dst(vector<int> v) {
        int n, k, sz = v.size();
        for (n = 1, k = 0; n < sz; n <<= 1, k++)
            ;
        t.assign(k, vector<int>(n));
        for (int i = 0; i < n; i++) {
            t[0][i] = i < sz ? v[i] : neutral;
        }
        for (int j = 0, len = 1; j <= k; j++, len <<= 1) {
            for (int s = len; s < n; s += (len << 1)) {
                t[j][s] = v[s];
                t[j][s - 1] = v[s - 1];
                for (int i = 1; i < len; i++) {
                    t[j][s + i] = comp(t[j][s + i - 1], v[s + i]);
                    t[j][s - 1 - i] = comp(v[s - 1 - i], t[j][s - i]);
                }
            }
        }
    }
    int query(int l, int r) {
        if (l == r) {
            return t[0][r];
        }
        int i = 31 - __builtin_clz(l ^ r);
        return comp(t[i][l], t[i][r]);
    }
};
\end{lstlisting}
\end{multicols}

\hfill

\subsection{Sparse Table}


*Read in [English](README.en.md)*



Responde consultas de maneira eficiente em um conjunto de dados estáticos.\

Realiza um pré-processamento para diminuir o tempo de cada consulta.



\begin{itemize}
\item Complexidade de tempo (Pré-processamento): O(N * log(N))
\item Complexidade de tempo (Consulta para operações sem sobreposição amigável): O(N * log(N))
\item Complexidade de tempo (Consulta para operações com sobreposição amigável): O(1)
\item Complexidade de espaço: O(N * log(N))
\end{itemize}



Exemplo de operações com sobreposição amigável: max(), min(), gcd(), f(x, y) = x





\hfill

\begin{multicols}{2}
\begin{lstlisting}[language=C++]
struct SparseTable {
    int n, e;
    vector<vector<int>> st;
    SparseTable(vector<int> &v) : n(v.size()), e(floor(log2(n))) {
        st.assign(e + 1, vector<int>(n));
        for (int i = 0; i < n; i++) {
            st[0][i] = v[i];
        }
        for (int i = 1; i <= e; i++) {
            for (int j = 0; j + (1 << i) <= n; j++) {
                st[i][j] = min(st[i - 1][j], st[i - 1][j + (1 << (i - 1))]);
            }
        }
    }
    // O(log(N)) Query for non overlap friendly
    // operations
    int logquery(int l, int r) {
        int res = 2e9;
        for (int i = e; i >= 0; i--) {
            if ((1 << i) <= r - l + 1) {
                res = min(res, st[i][l]);
                l += 1 << i;
            }
        }
        return res;
    }
    // O(1) Query for overlab friendly operations
    // ex: max(), min(), gcd(), f(x, y) = x
    int query(int l, int r) {
        // if (l > r) return 2e9;
        int i = ilogb(r - l + 1);
        return min(st[i][l], st[i][r - (1 << i) + 1]);
    }
};
\end{lstlisting}
\end{multicols}

\hfill

\section{Kd Fenwick Tree}


\textbf{KD Fenwick Tree} 



Fenwick Tree em K dimensoes.



* Complexidade de update: $O(log^k(N))$.

* Complexidade de query: $O(log^k(N))$.
\hfill

\begin{multicols}{2}
\begin{lstlisting}[language=C++]
const int MAX = 10;
ll tree[MAX][MAX][MAX][MAX][MAX][MAX][MAX][MAX]; // insira a quantidade
                                                 // necessaria de dimensoes

int lsONE(int x) {
    return x & (-x);
}

ll query(vector<int> s, int pos) {
    ll sum = 0;
    while (s[pos] > 0) {
        if (pos < s.size() - 1) {
            sum += query(s, pos + 1);
        } else {
            sum += tree[s[0]][s[1]][s[2]][s[3]][s[4]][s[5]][s[6]][s[7]];
        }
        s[pos] -= lsONE(s[pos]);
    }
    return sum;
}

void update(vector<int> s, int pos, int v) {
    while (s[pos] < MAX + 1) {
        if (pos < s.size() - 1) {
            update(s, pos + 1, v);
        } else {
            tree[s[0]][s[1]][s[2]][s[3]][s[4]][s[5]][s[6]][s[7]] += v;
        }

        s[pos] += lsONE(s[pos]);
    }
}\end{lstlisting}
\end{multicols}

\hfill

\section{Segment Tree}
\subsection{Segment Tree Beats Max And Sum Update}


Seg Tree que suporta update de maximo, update de soma e query de soma.

Utiliza uma fila de lazy para diferenciar os updates



\begin{itemize}
\item Complexidade de tempo (Pré-processamento): O(N)
\item Complexidade de tempo (Consulta em intervalo): O(log(N))
\item Complexidade de tempo (Update em ponto): O(log(N))
\item Complexidade de tempo (Update em intervalo): O(log(N))
\item Complexidade de espaço: 2 *4 *N = O(N)
\end{itemize}

\hfill

\begin{multicols}{2}
\begin{lstlisting}[language=C++]
#include <bits/stdc++.h>
using namespace std;

#define ll long long
#define INF 1e9
#define fi first
#define se second

typedef pair<int, int> ii;

struct Node {
    int m1 = INF, m2 = INF, cont = 0;
    ll soma = 0;
    queue<ii> lazy;

    void set(int v) {
        m1 = v;
        cont = 1;
        soma = v;
    }

    void merge(Node a, Node b) {
        m1 = min(a.m1, b.m1);
        m2 = INF;
        if (a.m1 != b.m1) {
            m2 = min(m2, max(a.m1, b.m1));
        }
        if (a.m2 != m1) {
            m2 = min(m2, a.m2);
        }
        if (b.m2 != m1) {
            m2 = min(m2, b.m2);
        }
        cont = (a.m1 == m1 ? a.cont : 0) + (b.m1 == m1 ? b.cont : 0);
        soma = a.soma + b.soma;
    }

    void print() {
        printf("%d %d %d %lld\n", m1, m2, cont, soma);
    }
};

int n, q;
vector<Node> tree;

int le(int n) {
    return 2 * n + 1;
}
int ri(int n) {
    return 2 * n + 2;
}

void push(int n, int esq, int dir) {
    while (!tree[n].lazy.empty()) {
        ii p = tree[n].lazy.front();
        tree[n].lazy.pop();
        int op = p.fi, v = p.se;
        if (op == 0) {
            if (v <= tree[n].m1) {
                continue;
            }
            tree[n].soma += (ll)abs(tree[n].m1 - v) * tree[n].cont;
            tree[n].m1 = v;
            if (esq != dir) {
                tree[le(n)].lazy.push({0, v});
                tree[ri(n)].lazy.push({0, v});
            }
        } else if (op == 1) {
            tree[n].soma += v * (dir - esq + 1);
            tree[n].m1 += v;
            tree[n].m2 += v;
            if (esq != dir) {
                tree[le(n)].lazy.push({1, v});
                tree[ri(n)].lazy.push({1, v});
            }
        }
    }
}

void build(int n, int esq, int dir, vector<int> &v) {
    if (esq == dir) {
        tree[n].set(v[esq]);
    } else {
        int mid = (esq + dir) / 2;
        build(le(n), esq, mid, v);
        build(ri(n), mid + 1, dir, v);
        tree[n].merge(tree[le(n)], tree[ri(n)]);
    }
}
void build(vector<int> &v) {
    build(0, 0, n - 1, v);
}

// ai = max(ai, mi) em [l, r]
void update(int n, int esq, int dir, int l, int r, int mi) {
    push(n, esq, dir);
    if (esq > r || dir < l || mi <= tree[n].m1) {
        return;
    }
    if (l <= esq && dir <= r && mi < tree[n].m2) {
        tree[n].soma += (ll)abs(tree[n].m1 - mi) * tree[n].cont;
        tree[n].m1 = mi;
        if (esq != dir) {
            tree[le(n)].lazy.push({0, mi});
            tree[ri(n)].lazy.push({0, mi});
        }
    } else {
        int mid = (esq + dir) / 2;
        update(le(n), esq, mid, l, r, mi);
        update(ri(n), mid + 1, dir, l, r, mi);
        tree[n].merge(tree[le(n)], tree[ri(n)]);
    }
}
void update(int l, int r, int mi) {
    update(0, 0, n - 1, l, r, mi);
}

// soma v em [l, r]
void upsoma(int n, int esq, int dir, int l, int r, int v) {
    push(n, esq, dir);
    if (esq > r || dir < l) {
        return;
    }
    if (l <= esq && dir <= r) {
        tree[n].soma += v * (dir - esq + 1);
        tree[n].m1 += v;
        tree[n].m2 += v;
        if (esq != dir) {
            tree[le(n)].lazy.push({1, v});
            tree[ri(n)].lazy.push({1, v});
        }
    } else {
        int mid = (esq + dir) / 2;
        upsoma(le(n), esq, mid, l, r, v);
        upsoma(ri(n), mid + 1, dir, l, r, v);
        tree[n].merge(tree[le(n)], tree[ri(n)]);
    }
}
void upsoma(int l, int r, int v) {
    upsoma(0, 0, n - 1, l, r, v);
}

// soma de [l, r]
int query(int n, int esq, int dir, int l, int r) {
    push(n, esq, dir);
    if (esq > r || dir < l) {
        return 0;
    }
    if (l <= esq && dir <= r) {
        return tree[n].soma;
    }
    int mid = (esq + dir) / 2;
    return query(le(n), esq, mid, l, r) + query(ri(n), mid + 1, dir, l, r);
}
int query(int l, int r) {
    return query(0, 0, n - 1, l, r);
}

int main() {
    cin >> n;
    tree.assign(4 * n, Node());
    build(v);
}
\end{lstlisting}
\end{multicols}

\hfill

\subsection{Segment Tree Esparsa}


Consultas e atualizações em intervalos.



\textbf{Seg Tree} 

Implementação padrão de Seg Tree



\begin{itemize}
\item Complexidade de tempo (Pré-processamento): O(N)
\item Complexidade de tempo (Consulta em intervalo): O(log(N))
\item Complexidade de tempo (Update em ponto): O(log(N))
\item Complexidade de espaço: 4 *N = O(N)
\end{itemize}



\textbf{Seg Tree Lazy} 

Implementação padrão de Seg Tree com lazy update



\begin{itemize}
\item Complexidade de tempo (Pré-processamento): O(N)
\item Complexidade de tempo (Consulta em intervalo): O(log(N))
\item Complexidade de tempo (Update em ponto): O(log(N))
\item Complexidade de tempo (Update em intervalo): O(log(N))
\item Complexidade de espaço: 2 *4 *N = O(N)
\end{itemize}



\textbf{Sparse Seg Tree} 

Seg Tree Esparsa:



\begin{itemize}
\item Complexidade de tempo (Pré-processamento): O(1)
\item Complexidade de tempo (Consulta em intervalo): O(log(N))
\item Complexidade de tempo (Update em ponto): O(log(N))
\end{itemize}



\textbf{Persistent Seg Tree} 

Seg Tree Esparsa com histórico de Updates:



\begin{itemize}
\item Complexidade de tempo (Pré-processamento): O(N *log(N))
\item Complexidade de tempo (Consulta em intervalo): O(log(N))
\item Complexidade de tempo (Update em ponto): O(log(N))
\item \textbf{Para fazer consulta em um tempo específico basta indicar o tempo na query}
\end{itemize}



\textbf{Seg Tree Beats} 

Seg Tree que suporta update de maximo e query de soma



\begin{itemize}
\item Complexidade de tempo (Pré-processamento): O(N)
\item Complexidade de tempo (Consulta em intervalo): O(log(N))
\item Complexidade de tempo (Update em ponto): O(log(N))
\item Complexidade de tempo (Update em intervalo): O(log(N))
\item Complexidade de espaço: 2 *4 *N = O(N)
\end{itemize}



\textbf{Seg Tree Beats Max and Sum update} 

Seg Tree que suporta update de maximo, update de soma e query de soma.

Utiliza uma fila de lazy para diferenciar os updates



\begin{itemize}
\item Complexidade de tempo (Pré-processamento): O(N)
\item Complexidade de tempo (Consulta em intervalo): O(log(N))
\item Complexidade de tempo (Update em ponto): O(log(N))
\item Complexidade de tempo (Update em intervalo): O(log(N))
\item Complexidade de espaço: 2 *4 *N = O(N)
\end{itemize}

\hfill

\begin{multicols}{2}
\begin{lstlisting}[language=C++]
const int SEGMAX = 8e6 + 5; // should be Q * log(DIR-ESQ+1)
const ll ESQ = 0, DIR = 1e9 + 7;

struct seg {
    ll tree[SEGMAX];
    int R[SEGMAX], L[SEGMAX],
        ptr = 2; // 0 is NULL; 1 is First Root
    ll op(ll a, ll b) {
        return (a + b) % MOD;
    }
    int le(int i) {
        if (L[i] == 0) {
            L[i] = ptr++;
        }
        return L[i];
    }
    int ri(int i) {
        if (R[i] == 0) {
            R[i] = ptr++;
        }
        return R[i];
    }
    ll query(ll l, ll r, int n = 1, ll esq = ESQ, ll dir = DIR) {
        if (r < esq || dir < l) {
            return 0;
        }
        if (l <= esq && dir <= r) {
            return tree[n];
        }
        ll mid = (esq + dir) / 2;
        return op(query(l, r, le(n), esq, mid), query(l, r, ri(n), mid + 1, dir));
    }
    void update(ll x, ll v, int n = 1, ll esq = ESQ, ll dir = DIR) {
        if (esq == dir) {
            tree[n] = (tree[n] + v) % MOD;
        } else {
            ll mid = (esq + dir) / 2;
            if (x <= mid) {
                update(x, v, le(n), esq, mid);
            } else {
                update(x, v, ri(n), mid + 1, dir);
            }
            tree[n] = op(tree[le(n)], tree[ri(n)]);
        }
    }
};
\end{lstlisting}
\end{multicols}

\hfill

\subsection{Segment Tree Beats Max Update}
Seg Tree que suporta update de maximo e query de soma



\begin{itemize}
\item Complexidade de tempo (Pré-processamento): O(N)
\item Complexidade de tempo (Consulta em intervalo): O(log(N))
\item Complexidade de tempo (Update em ponto): O(log(N))
\item Complexidade de tempo (Update em intervalo): O(log(N))
\item Complexidade de espaço: 2 *4 *N = O(N)
\end{itemize}

\hfill

\begin{multicols}{2}
\begin{lstlisting}[language=C++]
#include <bits/stdc++.h>
using namespace std;

#define ll long long
#define INF 1e9

struct Node {
    int m1 = INF, m2 = INF, cont = 0, lazy = 0;
    ll soma = 0;

    void set(int v) {
        m1 = v;
        cont = 1;
        soma = v;
    }

    void merge(Node a, Node b) {
        m1 = min(a.m1, b.m1);
        m2 = INF;
        if (a.m1 != b.m1) {
            m2 = min(m2, max(a.m1, b.m1));
        }
        if (a.m2 != m1) {
            m2 = min(m2, a.m2);
        }
        if (b.m2 != m1) {
            m2 = min(m2, b.m2);
        }
        cont = (a.m1 == m1 ? a.cont : 0) + (b.m1 == m1 ? b.cont : 0);
        soma = a.soma + b.soma;
    }

    void print() {
        printf("%d %d %d %lld %d\n", m1, m2, cont, soma, lazy);
    }
};

int n, q;
vector<Node> tree;

int le(int n) {
    return 2 * n + 1;
}
int ri(int n) {
    return 2 * n + 2;
}

void push(int n, int esq, int dir) {
    if (tree[n].lazy <= tree[n].m1) {
        return;
    }
    tree[n].soma += (ll)abs(tree[n].m1 - tree[n].lazy) * tree[n].cont;
    tree[n].m1 = tree[n].lazy;
    if (esq != dir) {
        tree[le(n)].lazy = max(tree[le(n)].lazy, tree[n].lazy);
        tree[ri(n)].lazy = max(tree[ri(n)].lazy, tree[n].lazy);
    }
    tree[n].lazy = 0;
}

void build(int n, int esq, int dir, vector<int> &v) {
    if (esq == dir) {
        tree[n].set(v[esq]);
    } else {
        int mid = (esq + dir) / 2;
        build(le(n), esq, mid, v);
        build(ri(n), mid + 1, dir, v);
        tree[n].merge(tree[le(n)], tree[ri(n)]);
    }
}
void build(vector<int> &v) {
    build(0, 0, n - 1, v);
}

// ai = max(ai, mi) em [l, r]
void update(int n, int esq, int dir, int l, int r, int mi) {
    push(n, esq, dir);
    if (esq > r || dir < l || mi <= tree[n].m1) {
        return;
    }
    if (l <= esq && dir <= r && mi < tree[n].m2) {
        tree[n].lazy = mi;
        push(n, esq, dir);
    } else {
        int mid = (esq + dir) / 2;
        update(le(n), esq, mid, l, r, mi);
        update(ri(n), mid + 1, dir, l, r, mi);
        tree[n].merge(tree[le(n)], tree[ri(n)]);
    }
}
void update(int l, int r, int mi) {
    update(0, 0, n - 1, l, r, mi);
}

// soma de [l, r]
int query(int n, int esq, int dir, int l, int r) {
    push(n, esq, dir);
    if (esq > r || dir < l) {
        return 0;
    }
    if (l <= esq && dir <= r) {
        return tree[n].soma;
    }
    int mid = (esq + dir) / 2;
    return query(le(n), esq, mid, l, r) + query(ri(n), mid + 1, dir, l, r);
}
int query(int l, int r) {
    return query(0, 0, n - 1, l, r);
}

int main() {
    cin >> n;
    tree.assign(4 * n, Node());
}
\end{lstlisting}
\end{multicols}

\hfill

\subsection{Segment Tree Kadani}


Implementação de uma Seg Tree que suporta update de soma e query de soma máxima em intervalo.



\begin{itemize}
\item Complexidade de tempo (Pré-processamento): O(N)
\item Complexidade de tempo (Consulta em intervalo): O(log(N))
\item Complexidade de tempo (Update em ponto): O(log(N))
\item Complexidade de espaço: 4 * N = O(N)
\end{itemize}

\hfill

\begin{multicols}{2}
\begin{lstlisting}[language=C++]
namespace seg {
    const int MAX = 1e5 + 5;
    struct node {
        ll pref, suff, sum, best;
    };
    node new_node(ll v) {
        return node{v, v, v, v};
    }
    const node NEUTRAL = {0, 0, 0, 0};
    node tree[4 * MAX];
    node merge(node a, node b) {
        ll pref = max(a.pref, a.sum + b.pref);
        ll suff = max(b.suff, b.sum + a.suff);
        ll sum = a.sum + b.sum;
        ll best = max(a.suff + b.pref, max(a.best, b.best));
        return node{pref, suff, sum, best};
    }

    int n;
    int le(int n) {
        return 2 * n + 1;
    }
    int ri(int n) {
        return 2 * n + 2;
    }
    void build(int n, int esq, int dir, const vector<ll> &v) {
        if (esq == dir) {
            tree[n] = new_node(v[esq]);
        } else {
            int mid = (esq + dir) / 2;
            build(le(n), esq, mid, v);
            build(ri(n), mid + 1, dir, v);
            tree[n] = merge(tree[le(n)], tree[ri(n)]);
        }
    }
    void build(const vector<ll> &v) {
        n = v.size();
        build(0, 0, n - 1, v);
    }
    node query(int n, int esq, int dir, int l, int r) {
        if (esq > r || dir < l) {
            return NEUTRAL;
        }
        if (l <= esq && dir <= r) {
            return tree[n];
        }
        int mid = (esq + dir) / 2;
        return merge(query(le(n), esq, mid, l, r), query(ri(n), mid + 1, dir, l, r));
    }
    ll query(int l, int r) {
        return query(0, 0, n - 1, l, r).best;
    }
    void update(int n, int esq, int dir, int x, ll v) {
        if (esq > x || dir < x) {
            return;
        }
        if (esq == dir) {
            tree[n] = new_node(v);
        } else {
            int mid = (esq + dir) / 2;
            if (x <= mid) {
                update(le(n), esq, mid, x, v);
            } else {
                update(ri(n), mid + 1, dir, x, v);
            }
            tree[n] = merge(tree[le(n)], tree[ri(n)]);
        }
    }
    void update(int x, ll v) {
        update(0, 0, n - 1, x, v);
    }
}
\end{lstlisting}
\end{multicols}

\hfill

\subsection{Segment Tree Persisente}


Seg Tree Esparsa com histórico de Updates:



\begin{itemize}
\item Complexidade de tempo (Pré-processamento): O(N *log(N))
\item Complexidade de tempo (Consulta em intervalo): O(log(N))
\item Complexidade de tempo (Update em ponto): O(log(N))
\item \textbf{Para fazer consulta em um tempo específico basta indicar o tempo na query}
\end{itemize}

\hfill

\begin{multicols}{2}
\begin{lstlisting}[language=C++]
namespace seg {
    const ll ESQ = 0, DIR = 1e9 + 7;
    struct node {
        ll v = 0;
        node *l = NULL, *r = NULL;
        node() {
        }
        node(ll v) : v(v) {
        }
        node(node *l, node *r) : l(l), r(r) {
            v = l->v + r->v;
        }
        void apply() {
            if (l == NULL) {
                l = new node();
            }
            if (r == NULL) {
                r = new node();
            }
        }
    };
    vector<node *> roots;
    void build() {
        roots.push_back(new node());
    }
    void push(node *n, int esq, int dir) {
        if (esq != dir) {
            n->apply();
        }
    }
    // sum v on x
    node *update(node *n, int esq, int dir, int x, int v) {
        push(n, esq, dir);
        if (esq == dir) {
            return new node(n->v + v);
        }
        int mid = (esq + dir) / 2;
        if (x <= mid) {
            return new node(update(n->l, esq, mid, x, v), n->r);
        } else {
            return new node(n->l, update(n->r, mid + 1, dir, x, v));
        }
    }
    int update(int root, int pos, int val) {
        node *novo = update(roots[root], ESQ, DIR, pos, val);
        roots.push_back(novo);
        return roots.size() - 1;
    }
    // sum in [L, R]
    ll query(node *n, int esq, int dir, int l, int r) {
        push(n, esq, dir);
        if (esq > r || dir < l) {
            return 0;
        }
        if (l <= esq && dir <= r) {
            return n->v;
        }
        int mid = (esq + dir) / 2;
        return query(n->l, esq, mid, l, r) + query(n->r, mid + 1, dir, l, r);
    }
    ll query(int root, int l, int r) {
        return query(roots[root], ESQ, DIR, l, r);
    }
    // kth min number in [L, R] (l_root can not be
    // 0)
    int kth(node *L, node *R, int esq, int dir, int k) {
        push(L, esq, dir);
        push(R, esq, dir);
        if (esq == dir) {
            return esq;
        }
        int mid = (esq + dir) / 2;
        int cont = R->l->v - L->l->v;
        if (cont >= k) {
            return kth(L->l, R->l, esq, mid, k);
        } else {
            return kth(L->r, R->r, mid + 1, dir, k - cont);
        }
    }
    int kth(int l_root, int r_root, int k) {
        return kth(roots[l_root - 1], roots[r_root], ESQ, DIR, k);
    }
};
\end{lstlisting}
\end{multicols}

\hfill

\subsection{Segment Tree 2D}


Segment Tree em 2 dimensões.



\begin{itemize}
\item Complexidade de tempo (Pré-processamento): O(N*M)
\item Complexidade de tempo (Consulta em intervalo): O(log(N)*log(M))
\item Complexidade de tempo (Update em ponto): O(log(N)*log(M))
\item Complexidade de espaço: 4 * N * 4 * M = O(N*M)
\end{itemize}

\hfill

\begin{multicols}{2}
\begin{lstlisting}[language=C++]
const int MAX = 2505;

int n, m, mat[MAX][MAX], tree[4 * MAX][4 * MAX];

int le(int x) {
    return 2 * x + 1;
}
int ri(int x) {
    return 2 * x + 2;
}

void build_y(int nx, int lx, int rx, int ny, int ly, int ry) {
    if (ly == ry) {
        if (lx == rx) {
            tree[nx][ny] = mat[lx][ly];
        } else {
            tree[nx][ny] = tree[le(nx)][ny] + tree[ri(nx)][ny];
        }
    } else {
        int my = (ly + ry) / 2;
        build_y(nx, lx, rx, le(ny), ly, my);
        build_y(nx, lx, rx, ri(ny), my + 1, ry);
        tree[nx][ny] = tree[nx][le(ny)] + tree[nx][ri(ny)];
    }
}
void build_x(int nx, int lx, int rx) {
    if (lx != rx) {
        int mx = (lx + rx) / 2;
        build_x(le(nx), lx, mx);
        build_x(ri(nx), mx + 1, rx);
    }
    build_y(nx, lx, rx, 0, 0, m - 1);
}
void build() {
    build_x(0, 0, n - 1);
}

void update_y(int nx, int lx, int rx, int ny, int ly, int ry, int x, int y, int v) {
    if (ly == ry) {
        if (lx == rx) {
            tree[nx][ny] = v;
        } else {
            tree[nx][ny] = tree[le(nx)][ny] + tree[ri(nx)][ny];
        }
    } else {
        int my = (ly + ry) / 2;
        if (y <= my) {
            update_y(nx, lx, rx, le(ny), ly, my, x, y, v);
        } else {
            update_y(nx, lx, rx, ri(ny), my + 1, ry, x, y, v);
        }
        tree[nx][ny] = tree[nx][le(ny)] + tree[nx][ri(ny)];
    }
}
void update_x(int nx, int lx, int rx, int x, int y, int v) {
    if (lx != rx) {
        int mx = (lx + rx) / 2;
        if (x <= mx) {
            update_x(le(nx), lx, mx, x, y, v);
        } else {
            update_x(ri(nx), mx + 1, rx, x, y, v);
        }
    }
    update_y(nx, lx, rx, 0, 0, m - 1, x, y, v);
}
void update(int x, int y, int v) {
    update_x(0, 0, n - 1, x, y, v);
}

int sum_y(int nx, int ny, int ly, int ry, int qly, int qry) {
    if (ry < qly || ly > qry) {
        return 0;
    }
    if (qly <= ly && ry <= qry) {
        return tree[nx][ny];
    }
    int my = (ly + ry) / 2;
    return sum_y(nx, le(ny), ly, my, qly, qry) + sum_y(nx, ri(ny), my + 1, ry, qly, qry);
}
int sum_x(int nx, int lx, int rx, int qlx, int qrx, int qly, int qry) {
    if (rx < qlx || lx > qrx) {
        return 0;
    }
    if (qlx <= lx && rx <= qrx) {
        return sum_y(nx, 0, 0, m - 1, qly, qry);
    }
    int mx = (lx + rx) / 2;
    return sum_x(le(nx), lx, mx, qlx, qrx, qly, qry) +
           sum_x(ri(nx), mx + 1, rx, qlx, qrx, qly, qry);
}
int sum(int lx, int rx, int ly, int ry) {
    return sum_x(0, 0, n - 1, lx, rx, ly, ry);
}
\end{lstlisting}
\end{multicols}

\hfill

\subsection{Segment Tree}
Implementação padrão de Seg Tree



\begin{itemize}
\item Complexidade de tempo (Pré-processamento): O(N)
\item Complexidade de tempo (Consulta em intervalo): O(log(N))
\item Complexidade de tempo (Update em ponto): O(log(N))
\item Complexidade de espaço: 4 *N = O(N)
\end{itemize}

\hfill

\begin{multicols}{2}
\begin{lstlisting}[language=C++]
namespace seg {
    const int MAX = 2e5 + 5;
    int n;
    ll tree[4 * MAX];
    ll merge(ll a, ll b) {
        return a + b;
    }
    int le(int n) {
        return 2 * n + 1;
    }
    int ri(int n) {
        return 2 * n + 2;
    }
    void build(int n, int esq, int dir, const vector<ll> &v) {
        if (esq == dir) {
            tree[n] = v[esq];
        } else {
            int mid = (esq + dir) / 2;
            build(le(n), esq, mid, v);
            build(ri(n), mid + 1, dir, v);
            tree[n] = merge(tree[le(n)], tree[ri(n)]);
        }
    }
    void build(const vector<ll> &v) {
        n = v.size();
        build(0, 0, n - 1, v);
    }
    ll query(int n, int esq, int dir, int l, int r) {
        if (esq > r || dir < l) {
            return 0;
        }
        if (l <= esq && dir <= r) {
            return tree[n];
        }
        int mid = (esq + dir) / 2;
        return merge(query(le(n), esq, mid, l, r), query(ri(n), mid + 1, dir, l, r));
    }
    ll query(int l, int r) {
        return query(0, 0, n - 1, l, r);
    }
    void update(int n, int esq, int dir, int x, ll v) {
        if (esq > x || dir < x) {
            return;
        }
        if (esq == dir) {
            tree[n] = v;
        } else {
            int mid = (esq + dir) / 2;
            if (x <= mid) {
                update(le(n), esq, mid, x, v);
            } else {
                update(ri(n), mid + 1, dir, x, v);
            }
            tree[n] = merge(tree[le(n)], tree[ri(n)]);
        }
    }
    void update(int x, ll v) {
        update(0, 0, n - 1, x, v);
    }
}
\end{lstlisting}
\end{multicols}

\hfill

\subsection{Segment Tree Lazy}


Implementação padrão de Seg Tree com lazy update



\begin{itemize}
\item Complexidade de tempo (Pré-processamento): O(N)
\item Complexidade de tempo (Consulta em intervalo): O(log(N))
\item Complexidade de tempo (Update em ponto): O(log(N))
\item Complexidade de tempo (Update em intervalo): O(log(N))
\item Complexidade de espaço: 2 * 4 * N = O(N)
\end{itemize}

\hfill

\begin{multicols}{2}
\begin{lstlisting}[language=C++]
namespace seg {
    const int MAX = 2e5 + 5;
    const ll NEUTRAL = 0; // merge(a, neutral) = a
    ll merge(ll a, ll b) {
        return a + b;
    }
    int sz; // size of the array
    ll tree[4 * MAX], lazy[4 * MAX];
    int le(int n) {
        return 2 * n + 1;
    }
    int ri(int n) {
        return 2 * n + 2;
    }
    void push(int n, int esq, int dir) {
        if (lazy[n] == 0) {
            return;
        }
        tree[n] += lazy[n] * (dir - esq + 1);
        if (esq != dir) {
            lazy[le(n)] += lazy[n];
            lazy[ri(n)] += lazy[n];
        }
        lazy[n] = 0;
    }
    void build(span<const ll> v, int n, int esq, int dir) {
        if (esq == dir) {
            tree[n] = v[esq];
        } else {
            int mid = (esq + dir) / 2;
            build(v, le(n), esq, mid);
            build(v, ri(n), mid + 1, dir);
            tree[n] = merge(tree[le(n)], tree[ri(n)]);
        }
    }
    void build(span<const ll> v) {
        sz = v.size();
        build(v, 0, 0, sz - 1);
    }
    ll query(int l, int r, int n = 0, int esq = 0, int dir = sz - 1) {
        push(n, esq, dir);
        if (esq > r || dir < l) {
            return NEUTRAL;
        }
        if (l <= esq && dir <= r) {
            return tree[n];
        }
        int mid = (esq + dir) / 2;
        return merge(query(l, r, le(n), esq, mid), query(l, r, ri(n), mid + 1, dir));
    }
    void update(int l, int r, ll v, int n = 0, int esq = 0, int dir = sz - 1) {
        push(n, esq, dir);
        if (esq > r || dir < l) {
            return;
        }
        if (l <= esq && dir <= r) {
            lazy[n] += v;
            push(n, esq, dir);
        } else {
            int mid = (esq + dir) / 2;
            update(l, r, v, le(n), esq, mid);
            update(l, r, v, ri(n), mid + 1, dir);
            tree[n] = merge(tree[le(n)], tree[ri(n)]);
        }
    }
}
\end{lstlisting}
\end{multicols}

\hfill

\newpage

%%%%%%
%
%
% Grafos
%
%
%%%%%%

\chapter{Grafos}

\section{Stoer–Wagner Min Cut}


Algortimo de Stoer-Wagner para encontrar o corte mínimo de um grafo.



O algoritmo de Stoer-Wagner é um algoritmo para resolver o problema de corte mínimo em grafos não direcionados com pesos não negativos. A ideia essencial deste algoritmo é encolher o grafo mesclando os vértices mais intensos até que o grafo contenha apenas dois conjuntos de vértices combinados



Complexidade de tempo: $O(V^3)$

\hfill

\begin{multicols}{2}
\begin{lstlisting}[language=C++]
const int MAXN = 555, INF = 1e9 + 7;

int n, e, adj[MAXN][MAXN];
vector<int> bestCut;

int mincut() {
    int bestCost = INF;
    vector<int> v[MAXN];
    for (int i = 0; i < n; i++) {
        v[i].assign(1, i);
    }
    int w[MAXN], sel;
    bool exist[MAXN], added[MAXN];
    memset(exist, true, sizeof(exist));
    for (int phase = 0; phase < n - 1; phase++) {
        memset(added, false, sizeof(added));
        memset(w, 0, sizeof(w));
        for (int j = 0, prev; j < n - phase; j++) {
            sel = -1;
            for (int i = 0; i < n; i++) {
                if (exist[i] && !added[i] && (sel == -1 || w[i] > w[sel])) {
                    sel = i;
                }
            }
            if (j == n - phase - 1) {
                if (w[sel] < bestCost) {
                    bestCost = w[sel];
                    bestCut = v[sel];
                }
                v[prev].insert(v[prev].end(), v[sel].begin(), v[sel].end());
                for (int i = 0; i < n; i++) {
                    adj[prev][i] = adj[i][prev] += adj[sel][i];
                }
                exist[sel] = false;
            } else {
                added[sel] = true;
                for (int i = 0; i < n; i++) {
                    w[i] += adj[sel][i];
                }
                prev = sel;
            }
        }
    }
    return bestCost;
}\end{lstlisting}
\end{multicols}

\hfill

\section{Shortest Paths}
\subsection{Dijkstra}


Computa o menor caminho entre nós de um grafo.



\textbf{} 


Dado dois nós u e v, computa o menor caminho de u para v.



Complexidade de tempo: O((E + V) * log(E))



\textbf{} 


Dado um nó u, computa o menor caminho de u para todos os nós.



Complexidade de tempo: O((E + V) * log(E))



\textbf{} 


Computa o menor caminho de todos os nós para todos os nós



Complexidade de tempo: O(V * ((E + V) * log(E)))

\hfill

\begin{multicols}{2}
\begin{lstlisting}[language=C++]
const int MAX = 1e5 + 5, INF = 1e9 + 9;

vector<ii> adj[MAX];
int dist[MAX];

int dk(int s, int t) {
    priority_queue<ii, vector<ii>, greater<ii>> fila;
    fill(begin(dist), end(dist), INF);
    dist[s] = 0;
    fila.emplace(dist[s], s);
    while (!fila.empty()) {
        auto [d, u] = fila.top();
        fila.pop();
        if (u == t) {
            return dist[t];
        }
        if (d != dist[u]) {
            continue;
        }
        for (auto [w, v] : adj[u]) {
            if (dist[v] > d + w) {
                dist[v] = d + w;
                fila.emplace(dist[v], v);
            }
        }
    }
    return -1;
}
\end{lstlisting}
\end{multicols}

\hfill

\begin{multicols}{2}
\begin{lstlisting}[language=C++]
const int MAX = 505, INF = 1e9 + 9;

vector<ii> adj[MAX];
int dist[MAX][MAX];

void dk(int n) {
    for (int i = 0; i < n; i++) {
        for (int j = 0; j < n; j++) {
            dist[i][j] = INF;
        }
    }
    for (int s = 0; s < n; s++) {
        priority_queue<ii, vector<ii>, greater<ii>> fila;
        dist[s][s] = 0;
        fila.emplace(dist[s][s], s);
        while (!fila.empty()) {
            auto [d, u] = fila.top();
            fila.pop();
            if (d != dist[s][u]) {
                continue;
            }
            for (auto [w, v] : adj[u]) {
                if (dist[s][v] > d + w) {
                    dist[s][v] = d + w;
                    fila.emplace(dist[s][v], v);
                }
            }
        }
    }
}
\end{lstlisting}
\end{multicols}

\hfill

\begin{multicols}{2}
\begin{lstlisting}[language=C++]
const int MAX = 1e5 + 5, INF = 1e9 + 9;

vector<ii> adj[MAX];
int dist[MAX];

void dk(int s) {
    priority_queue<ii, vector<ii>, greater<ii>> fila;
    fill(begin(dist), end(dist), INF);
    dist[s] = 0;
    fila.emplace(dist[s], s);
    while (!fila.empty()) {
        auto [d, u] = fila.top();
        fila.pop();
        if (d != dist[u]) {
            continue;
        }
        for (auto [w, v] : adj[u]) {
            if (dist[v] > d + w) {
                dist[v] = d + w;
                fila.emplace(dist[v], v);
            }
        }
    }
}
\end{lstlisting}
\end{multicols}

\hfill

\subsection{SPFA}


Encontra o caminho mais curto entre um vértice e todos os outros vértices de um grafo.



Detecta ciclos negativos.



Complexidade de tempo: O(|V| * |E|)

\hfill

\begin{multicols}{2}
\begin{lstlisting}[language=C++]
const int MAX = 1e4 + 4;
const ll INF = 1e18 + 18;

vector<ii> adj[MAX];
ll dist[MAX];

void spfa(int s, int n) {
    fill(dist, dist + n, INF);
    vector<int> cnt(n, 0);
    vector<bool> inq(n, false);
    queue<int> fila;
    fila.push(s);
    inq[s] = true;
    dist[s] = 0;
    while (!fila.empty()) {
        int u = fila.front();
        fila.pop();
        inq[u] = false;
        for (auto [w, v] : adj[u]) {
            ll newd = (dist[u] == -INF ? -INF : max(w + dist[u], -INF));
            if (newd < dist[v]) {
                dist[v] = newd;
                if (!inq[v]) {
                    fila.push(v);
                    inq[v] = true;
                    cnt[v]++;
                    if (cnt[v] > n) { // negative cycle
                        dist[v] = -INF;
                    }
                }
            }
        }
    }
}
\end{lstlisting}
\end{multicols}

\hfill

\section{Inverse Graph}


Algoritmo que encontra as componentes conexas quando se é dado o grafo complemento.



Resolve problemas em que se deseja encontrar as componentes conexas quando são dadas as arestas que não pertencem ao grafo



\begin{itemize}
\item Complexidade de tempo: O(N log N + N log M)
\end{itemize}

\hfill

\begin{multicols}{2}
\begin{lstlisting}[language=C++]
#include <bits/stdc++.h>
using namespace std;

set<int> nodes;
vector<set<int>> adj;

void bfs(int s) {
    queue<int> f;
    f.push(s);
    nodes.erase(s);
    set<int> aux;
    while (!f.empty()) {
        int x = f.front();
        f.pop();
        for (int y : nodes) {
            if (adj[x].count(y) == 0) {
                aux.insert(y);
            }
        }
        for (int y : aux) {
            f.push(y);
            nodes.erase(y);
        }
        aux.clear();
    }
}
\end{lstlisting}
\end{multicols}

\hfill

\section{2 SAT}


Resolve problema do 2-SAT.



\begin{itemize}
\item Complexidade de tempo (caso médio): O(N + M)
\end{itemize}



N é o número de variáveis e M é o número de cláusulas.  

A configuração da solução fica guardada no vetor *assignment*. 



Em relaçõa ao sinal, tanto faz se 0 liga ou desliga, apenas siga o mesmo padrão.

\hfill

\begin{multicols}{2}
\begin{lstlisting}[language=C++]
struct sat2 {
    int n;
    vector<vector<int>> g, gt;
    vector<bool> used;
    vector<int> order, comp;
    vector<bool> assignment;

    // number of variables
    sat2(int _n) {
        n = 2 * (_n + 5);
        g.assign(n, vector<int>());
        gt.assign(n, vector<int>());
    }
    void add_edge(int v, int u, bool v_sign, bool u_sign) {
        g[2 * v + v_sign].push_back(2 * u + !u_sign);
        g[2 * u + u_sign].push_back(2 * v + !v_sign);
        gt[2 * u + !u_sign].push_back(2 * v + v_sign);
        gt[2 * v + !v_sign].push_back(2 * u + u_sign);
    }
    void dfs1(int v) {
        used[v] = true;
        for (int u : g[v]) {
            if (!used[u]) {
                dfs1(u);
            }
        }
        order.push_back(v);
    }
    void dfs2(int v, int cl) {
        comp[v] = cl;
        for (int u : gt[v]) {
            if (comp[u] == -1) {
                dfs2(u, cl);
            }
        }
    }
    bool solve() {
        order.clear();
        used.assign(n, false);
        for (int i = 0; i < n; ++i) {
            if (!used[i]) {
                dfs1(i);
            }
        }

        comp.assign(n, -1);
        for (int i = 0, j = 0; i < n; ++i) {
            int v = order[n - i - 1];
            if (comp[v] == -1) {
                dfs2(v, j++);
            }
        }

        assignment.assign(n / 2, false);
        for (int i = 0; i < n; i += 2) {
            if (comp[i] == comp[i + 1]) {
                return false;
            }
            assignment[i / 2] = comp[i] > comp[i + 1];
        }
        return true;
    }
};
\end{lstlisting}
\end{multicols}

\hfill

\section{Fluxo}


Conjunto de algoritmos para calcular o fluxo máximo em redes de fluxo.



\textbf{} 


Muito útil para grafos bipartidos e para grafos com muitas arestas



Complexidade de tempo: O(V² * E), mas em grafo bipartido a complexidade é O(sqrt(V) * E)



\textbf{} 


Útil para grafos com poucas arestas



Complexidade de tempo: O(V * E²)



\textbf{} 


Computa o fluxo máximo com custo mínimo



Complexidade de tempo: O(V² * E²)

\hfill

\begin{multicols}{2}
\begin{lstlisting}[language=C++]
const long long INF = 1e18;

struct FlowEdge {
    int u, v;
    long long cap, flow = 0;
    FlowEdge(int u, int v, long long cap) : u(u), v(v), cap(cap) {
    }
};

struct EdmondsKarp {
    int n, s, t, m = 0, vistoken = 0;
    vector<FlowEdge> edges;
    vector<vector<int>> adj;
    vector<int> visto;

    EdmondsKarp(int n, int s, int t) : n(n), s(s), t(t) {
        adj.resize(n);
        visto.resize(n);
    }

    void add_edge(int u, int v, long long cap) {
        edges.emplace_back(u, v, cap);
        edges.emplace_back(v, u, 0);
        adj[u].push_back(m);
        adj[v].push_back(m + 1);
        m += 2;
    }

    int bfs() {
        vistoken++;
        queue<int> fila;
        fila.push(s);
        vector<int> pego(n, -1);
        while (!fila.empty()) {
            int u = fila.front();
            if (u == t) {
                break;
            }
            fila.pop();
            visto[u] = vistoken;
            for (int id : adj[u]) {
                if (edges[id].cap - edges[id].flow < 1) {
                    continue;
                }
                int v = edges[id].v;
                if (visto[v] == -1) {
                    continue;
                }
                fila.push(v);
                pego[v] = id;
            }
        }
        if (pego[t] == -1) {
            return 0;
        }
        long long f = INF;
        for (int id = pego[t]; id != -1; id = pego[edges[id].u]) {
            f = min(f, edges[id].cap - edges[id].flow);
        }
        for (int id = pego[t]; id != -1; id = pego[edges[id].u]) {
            edges[id].flow += f;
            edges[id ^ 1].flow -= f;
        }
        return f;
    }

    long long flow() {
        long long maxflow = 0;
        while (long long f = bfs()) {
            maxflow += f;
        }
        return maxflow;
    }
};
\end{lstlisting}
\end{multicols}

\hfill

\begin{multicols}{2}
\begin{lstlisting}[language=C++]
typedef long long ll;

const ll INF = 1e18;

struct FlowEdge {
    int u, v;
    ll cap, flow = 0;
    FlowEdge(int u, int v, ll cap) : u(u), v(v), cap(cap) {
    }
};

struct Dinic {
    vector<FlowEdge> edges;
    vector<vector<int>> adj;
    int n, s, t, m = 0;
    vector<int> level, ptr;
    queue<int> q;

    Dinic(int n, int s, int t) : n(n), s(s), t(t) {
        adj.resize(n);
        level.resize(n);
        ptr.resize(n);
    }

    void add_edge(int u, int v, ll cap) {
        edges.emplace_back(u, v, cap);
        edges.emplace_back(v, u, 0);
        adj[u].push_back(m);
        adj[v].push_back(m + 1);
        m += 2;
    }

    bool bfs() {
        while (!q.empty()) {
            int u = q.front();
            q.pop();
            for (int id : adj[u]) {
                if (edges[id].cap - edges[id].flow < 1) {
                    continue;
                }
                int v = edges[id].v;
                if (level[v] != -1) {
                    continue;
                }
                level[v] = level[u] + 1;
                q.push(v);
            }
        }
        return level[t] != -1;
    }

    ll dfs(int u, ll f) {
        if (f == 0) {
            return 0;
        }
        if (u == t) {
            return f;
        }
        for (int &cid = ptr[u]; cid < (int)adj[u].size(); cid++) {
            int id = adj[u][cid];
            int v = edges[id].v;
            if (level[u] + 1 != level[v] || edges[id].cap - edges[id].flow < 1) {
                continue;
            }
            ll tr = dfs(v, min(f, edges[id].cap - edges[id].flow));
            if (tr == 0) {
                continue;
            }
            edges[id].flow += tr;
            edges[id ^ 1].flow -= tr;
            return tr;
        }
        return 0;
    }

    ll flow() {
        ll maxflow = 0;
        while (true) {
            fill(level.begin(), level.end(), -1);
            level[s] = 0;
            q.push(s);
            if (!bfs()) {
                break;
            }
            fill(ptr.begin(), ptr.end(), 0);
            while (ll f = dfs(s, INF)) {
                maxflow += f;
            }
        }
        return maxflow;
    }
};
\end{lstlisting}
\end{multicols}

\hfill

\begin{multicols}{2}
\begin{lstlisting}[language=C++]
struct MinCostMaxFlow {
    int n, s, t, m = 0;
    ll maxflow = 0, mincost = 0;
    vector<FlowEdge> edges;
    vector<vector<int>> adj;

    MinCostMaxFlow(int n, int s, int t) : n(n), s(s), t(t) {
        adj.resize(n);
    }

    void add_edge(int u, int v, ll cap, ll cost) {
        edges.emplace_back(u, v, cap, cost);
        edges.emplace_back(v, u, 0, -cost);
        adj[u].push_back(m);
        adj[v].push_back(m + 1);
        m += 2;
    }

    bool spfa() {
        vector<int> pego(n, -1);
        vector<ll> dis(n, INF);
        vector<bool> inq(n, false);
        queue<int> fila;
        fila.push(s);
        dis[s] = 0;
        inq[s] = 1;
        while (!fila.empty()) {
            int u = fila.front();
            fila.pop();
            inq[u] = false;
            for (int id : adj[u]) {
                if (edges[id].cap - edges[id].flow < 1) {
                    continue;
                }
                int v = edges[id].v;
                if (dis[v] > dis[u] + edges[id].cost) {
                    dis[v] = dis[u] + edges[id].cost;
                    pego[v] = id;
                    if (!inq[v]) {
                        inq[v] = true;
                        fila.push(v);
                    }
                }
            }
        }

        if (pego[t] == -1) {
            return 0;
        }
        ll f = INF;
        for (int id = pego[t]; id != -1; id = pego[edges[id].u]) {
            f = min(f, edges[id].cap - edges[id].flow);
            mincost += edges[id].cost;
        }
        for (int id = pego[t]; id != -1; id = pego[edges[id].u]) {
            edges[id].flow += f;
            edges[id ^ 1].flow -= f;
        }
        maxflow += f;
        return 1;
    }

    ll flow() {
        while (spfa())
            ;
        return maxflow;
    }
};
\end{lstlisting}
\end{multicols}

\hfill

\section{Kruskal}


Algoritimo para encontrar a MST (minimum spanning tree) de um grafo.



Utiliza [DSU](../../Estruturas\%20de\%20Dados/DSU/dsu.cpp) - (disjoint set union) - para construir MST - (minimum spanning tree)



\begin{itemize}
\item Complexidade de tempo (Construção): O(M log N)
\end{itemize}

\hfill

\begin{multicols}{2}
\begin{lstlisting}[language=C++]
struct Edge {
    int u, v, w;
    bool operator<(Edge const &other) {
        return w < other.w;
    }
};

vector<Edge> edges, result;
int cost;

struct DSU {
    vector<int> pa, sz;
    DSU(int n) {
        sz.assign(n + 5, 1);
        for (int i = 0; i < n + 5; i++) {
            pa.push_back(i);
        }
    }
    int root(int a) {
        return pa[a] = (a == pa[a] ? a : root(pa[a]));
    }
    bool find(int a, int b) {
        return root(a) == root(b);
    }
    void uni(int a, int b) {
        int ra = root(a), rb = root(b);
        if (ra == rb) {
            return;
        }
        if (sz[ra] > sz[rb]) {
            swap(ra, rb);
        }
        pa[ra] = rb;
        sz[rb] += sz[ra];
    }
};

void kruskal(int m, int n) {
    DSU dsu(n);

    sort(edges.begin(), edges.end());

    for (Edge e : edges) {
        if (!dsu.find(e.u, e.v)) {
            cost += e.w;
            result.push_back(e); // remove if need only cost
            dsu.uni(e.u, e.v);
        }
    }
}
\end{lstlisting}
\end{multicols}

\hfill

\section{Graph Center}


Encontra o centro e o diâmetro de um grafo



Complexidade de tempo: O(N)

\hfill

\begin{multicols}{2}
\begin{lstlisting}[language=C++]
const int INF = 1e9 + 9;

vector<vector<int>> adj;

struct GraphCenter {
    int n, diam = 0;
    vector<int> centros, dist, pai;
    int bfs(int s) {
        queue<int> q;
        q.push(s);
        dist.assign(n + 5, INF);
        pai.assign(n + 5, -1);
        dist[s] = 0;
        int maxidist = 0, maxinode = 0;
        while (!q.empty()) {
            int u = q.front();
            q.pop();
            if (dist[u] >= maxidist) {
                maxidist = dist[u], maxinode = u;
            }
            for (int v : adj[u]) {
                if (dist[u] + 1 < dist[v]) {
                    dist[v] = dist[u] + 1;
                    pai[v] = u;
                    q.push(v);
                }
            }
        }
        diam = max(diam, maxidist);
        return maxinode;
    }
    GraphCenter(int st = 0) : n(adj.size()) {
        int d1 = bfs(st);
        int d2 = bfs(d1);
        vector<int> path;
        for (int u = d2; u != -1; u = pai[u]) {
            path.push_back(u);
        }
        int len = path.size();
        if (len % 2 == 1) {
            centros.push_back(path[len / 2]);
        } else {
            centros.push_back(path[len / 2]);
            centros.push_back(path[len / 2 - 1]);
        }
    }
};
\end{lstlisting}
\end{multicols}

\hfill

\section{Bridge}


Algoritmo que acha pontes utilizando uma dfs



Complexidade de tempo: O(N + M)
\hfill

\begin{multicols}{2}
\begin{lstlisting}[language=C++]
int n;                   // number of nodes
vector<vector<int>> adj; // adjacency list of graph

vector<bool> visited;
vector<int> tin, low;
int timer;

void dfs(int u, int p = -1) {
    visited[u] = true;
    tin[u] = low[u] = timer++;
    for (int v : adj[u]) {
        if (v == p) {
            continue;
        }
        if (visited[v]) {
            low[u] = min(low[u], tin[v]);
        } else {
            dfs(v, u);
            low[u] = min(low[u], low[v]);
            if (low[v] > tin[u]) {
                // edge UV is a bridge
                // do_something(u, v)
            }
        }
    }
}

void find_bridges() {
    timer = 0;
    visited.assign(n, false);
    tin.assign(n, -1);
    low.assign(n, -1);
    for (int i = 0; i < n; ++i) {
        if (!visited[i]) {
            dfs(i);
        }
    }
}
\end{lstlisting}
\end{multicols}

\hfill

\section{HLD}


Técnica usada para otimizar a execução de operações em árvores.



\begin{itemize}
\item Pré-Processamento: O(N)
\item Range Query/Update: O(Log(N)) * O(Complexidade de query da estrutura)
\item Point Query/Update: O(Complexidade de query da estrutura)
\item LCA: O(Log(N))
\item Subtree Query: O(Complexidade de query da estrutura)
\item Complexidade de espaço: O(N)
\end{itemize}



\hfill

\begin{multicols}{2}
\begin{lstlisting}[language=C++]
namespace hld {
    const int MAX = 2e5 + 5;
    int t, sz[MAX], pos[MAX], pai[MAX], head[MAX];
    bool e = 0;
    ll merge(ll a, ll b) {
        return max(a, b);
    } // how to merge paths
    void dfs_sz(int u, int p = -1) {
        sz[u] = 1;
        for (int &v : adj[u]) {
            if (v != p) {
                dfs_sz(v, u);
                sz[u] += sz[v];
                if (sz[v] > sz[adj[u][0]] || adj[u][0] == p) {
                    swap(v, adj[u][0]);
                }
            }
        }
    }
    void dfs_hld(int u, int p = -1) {
        pos[u] = t++;
        for (int v : adj[u]) {
            if (v != p) {
                pai[v] = u;
                head[v] = (v == adj[u][0] ? head[u] : v);
                dfs_hld(v, u);
            }
        }
    }
    void build(int root) {
        dfs_sz(root);
        t = 0;
        pai[root] = root;
        head[root] = root;
        dfs_hld(root);
    }
    void build(int root, vector<ll> &v) {
        build(root);
        vector<ll> aux(v.size());
        for (int i = 0; i < (int)v.size(); i++) {
            aux[pos[i]] = v[i];
        }
        seg::build(aux);
    }
    void build(int root,
               vector<i3> &edges) { // use this if
                                    // weighted edges
        build(root);
        e = 1;
        vector<ll> aux(edges.size() + 1);
        for (auto [u, v, w] : edges) {
            if (pos[u] > pos[v]) {
                swap(u, v);
            }
            aux[pos[v]] = w;
        }
        seg::build(aux);
    }
    ll query(int u, int v) {
        if (pos[u] > pos[v]) {
            swap(u, v);
        }
        if (head[u] == head[v]) {
            return seg::query(pos[u] + e, pos[v]);
        } else {
            ll qv = seg::query(pos[head[v]], pos[v]);
            ll qu = query(u, pai[head[v]]);
            return merge(qu, qv);
        }
    }
    void update(int u, int v, ll k) {
        if (pos[u] > pos[v]) {
            swap(u, v);
        }
        if (head[u] == head[v]) {
            seg::update(pos[u] + e, pos[v], k);
        } else {
            seg::update(pos[head[v]], pos[v], k);
            update(u, pai[head[v]], k);
        }
    }
    int lca(int u, int v) {
        if (pos[u] > pos[v]) {
            swap(u, v);
        }
        return (head[u] == head[v] ? u : lca(u, pai[head[v]]));
    }
    ll query_subtree(int u) {
        return seg::query(pos[u], pos[u] + sz[u] - 1);
    }
}
\end{lstlisting}
\end{multicols}

\hfill

\section{Matching}
\subsection{Hungaro}


Resolve o problema de Matching para uma matriz \lstinline{A[n][m]}, onde $n \leq m$.



A implementação minimiza os custos, para maximizar basta multiplicar os pesos por -1.



\textbf{A matriz de entrada precisa ser indexada em 1 !!!}



O vetor \lstinline{result} guarda os pares do matching.



Complexidade de tempo: $O(n^2 * m)$
\hfill

\begin{multicols}{2}
\begin{lstlisting}[language=C++]
const ll INF = 1e18 + 18;

vector<pair<int, int>> result;

ll hungarian(int n, int m, vector<vector<int>> &A) {
    vector<int> u(n + 1), v(m + 1), p(m + 1), way(m + 1);
    for (int i = 1; i <= n; i++) {
        p[0] = i;
        int j0 = 0;
        vector<int> minv(m + 1, INF);
        vector<char> used(m + 1, false);
        do {
            used[j0] = true;
            ll i0 = p[j0], delta = INF, j1;
            for (int j = 1; j <= m; j++) {
                if (!used[j]) {
                    int cur = A[i0][j] - u[i0] - v[j];
                    if (cur < minv[j]) {
                        minv[j] = cur, way[j] = j0;
                    }
                    if (minv[j] < delta) {
                        delta = minv[j], j1 = j;
                    }
                }
            }
            for (int j = 0; j <= m; j++) {
                if (used[j]) {
                    u[p[j]] += delta, v[j] -= delta;
                } else {
                    minv[j] -= delta;
                }
            }
            j0 = j1;
        } while (p[j0] != 0);
        do {
            int j1 = way[j0];
            p[j0] = p[j1];
            j0 = j1;
        } while (j0);
    }
    for (int i = 1; i <= m; i++) {
        result.emplace_back(p[i], i);
    }
    return -v[0];
}
\end{lstlisting}
\end{multicols}

\hfill

\section{Binary Lifting}


Usa uma sparse table para calcular o k-ésimo ancestral de u.



Pode ser usada com o algoritmo de EulerTour para calcular o LCA.



Complexidade de tempo:



\begin{itemize}
\item Pré-processamento: O(N * log(N))
\item Consulta do k-ésimo ancestral de u: O(log(N))
\item LCA: O(log(N))
\end{itemize}



Complexidade de espaço: O(Nlog(N))

\hfill

\begin{multicols}{2}
\begin{lstlisting}[language=C++]
namespace st {
    int n, me;
    vector<vector<int>> st;
    void bl_dfs(int u, int p) {
        st[u][0] = p;
        for (int i = 1; i <= me; i++) {
            st[u][i] = st[st[u][i - 1]][i - 1];
        }
        for (int v : adj[u]) {
            if (v != p) {
                bl_dfs(v, u);
            }
        }
    }
    void build(int _n, int root = 0) {
        n = _n;
        me = floor(log2(n));
        st.assign(n, vector<int>(me + 1, 0));
        bl_dfs(root, root);
    }
    int ancestor(int u,
                 int k) { // k-th ancestor of u
        for (int i = me; i >= 0; i--) {
            if ((1 << i) & k) {
                u = st[u][i];
            }
        }
        return u;
    }
}
\end{lstlisting}
\end{multicols}

\hfill

\begin{multicols}{2}
\begin{lstlisting}[language=C++]
namespace st {
    int n, me, timer;
    vector<int> tin, tout;
    vector<vector<int>> st;
    void et_dfs(int u, int p) {
        tin[u] = ++timer;
        st[u][0] = p;
        for (int i = 1; i <= me; i++) {
            st[u][i] = st[st[u][i - 1]][i - 1];
        }
        for (int v : adj[u]) {
            if (v != p) {
                et_dfs(v, u);
            }
        }
        tout[u] = ++timer;
    }
    void build(int _n, int root = 0) {
        n = _n;
        tin.assign(n, 0);
        tout.assign(n, 0);
        timer = 0;
        me = floor(log2(n));
        st.assign(n, vector<int>(me + 1, 0));
        et_dfs(root, root);
    }
    bool is_ancestor(int u, int v) {
        return tin[u] <= tin[v] && tout[u] >= tout[v];
    }
    int lca(int u, int v) {
        if (is_ancestor(u, v)) {
            return u;
        }
        if (is_ancestor(v, u)) {
            return v;
        }
        for (int i = me; i >= 0; i--) {
            if (!is_ancestor(st[u][i], v)) {
                u = st[u][i];
            }
        }
        return st[u][0];
    }
    int ancestor(int u,
                 int k) { // k-th ancestor of u
        for (int i = me; i >= 0; i--) {
            if ((1 << i) & k) {
                u = st[u][i];
            }
        }
        return u;
    }
}
\end{lstlisting}
\end{multicols}

\hfill

\section{LCA}


Algoritmo de Lowest Common Ancestor usando EulerTour e Sparse Table



Complexidade de tempo:



\begin{itemize}
\item O(Nlog(N)) Preprocessing
\item O(1) Query LCA
\end{itemize}



Complexidade de espaço: O(Nlog(N))

\hfill

\begin{multicols}{2}
\begin{lstlisting}[language=C++]
#include <bits/stdc++.h>
using namespace std;

#define INF 1e9
#define fi first
#define se second

typedef pair<int, int> ii;

vector<int> tin, tout;
vector<vector<int>> adj;
vector<ii> prof;
vector<vector<ii>> st;

int n, timer;

void SparseTable(vector<ii> &v) {
    int n = v.size();
    int e = floor(log2(n));
    st.assign(e + 1, vector<ii>(n));
    for (int i = 0; i < n; i++) {
        st[0][i] = v[i];
    }
    for (int i = 1; i <= e; i++) {
        for (int j = 0; j + (1 << i) <= n; j++) {
            st[i][j] = min(st[i - 1][j], st[i - 1][j + (1 << (i - 1))]);
        }
    }
}

void et_dfs(int u, int p, int h) {
    tin[u] = timer++;
    prof.emplace_back(h, u);
    for (int v : adj[u]) {
        if (v != p) {
            et_dfs(v, u, h + 1);
            prof.emplace_back(h, u);
        }
    }
    tout[u] = timer++;
}

void build(int root = 0) {
    tin.assign(n, 0);
    tout.assign(n, 0);
    prof.clear();
    timer = 0;
    et_dfs(root, root, 0);
    SparseTable(prof);
}

int lca(int u, int v) {
    int l = tout[u], r = tin[v];
    if (l > r) {
        swap(l, r);
    }
    int i = floor(log2(r - l + 1));
    return min(st[i][l], st[i][r - (1 << i) + 1]).se;
}

int main() {
    cin >> n;

    adj.assign(n, vector<int>(0));

    for (int i = 0; i < n - 1; i++) {
        int a, b;
        cin >> a >> b;
        adj[a].push_back(b);
        adj[b].push_back(a);
    }

    build();
}
\end{lstlisting}
\end{multicols}

\hfill

\newpage

%%%%%%
%
%
% String
%
%
%%%%%%

\chapter{String}

\section{Aho Corasick}


Constrói uma estrutura de dados semelhante a um trie com links adicionais e, em seguida, constrói uma máquina de estados finitos (autômato). Útil para pattern matching de um set de strings em um texto.



Complexidade de tempo: O(|S|+|T|), onde |S| é o somatório do tamanho das strings e |T| é o tamanho do texto

\hfill

\begin{multicols}{2}
\begin{lstlisting}[language=C++]
const int K = 26;

struct Vertex {
    int next[K], p = -1, link = -1, exi = -1, go[K], cont = 0;
    bool term = false;
    vector<int> idxs;
    char pch;
    Vertex(int p = -1, char ch = '$') : p(p), pch(ch) {
        fill(begin(next), end(next), -1);
        fill(begin(go), end(go), -1);
    }
};
vector<Vertex> aho(1);
void add_string(const string &s, int idx) {
    int v = 0;
    for (char ch : s) {
        int c = ch - 'a';
        if (aho[v].next[c] == -1) {
            aho[v].next[c] = aho.size();
            aho.emplace_back(v, ch);
        }
        v = aho[v].next[c];
    }
    aho[v].term = true;
    aho[v].idxs.push_back(idx);
}
int go(int u, char ch);
int get_link(int u) {
    if (aho[u].link == -1) {
        if (u == 0 || aho[u].p == 0) {
            aho[u].link = 0;
        } else {
            aho[u].link = go(get_link(aho[u].p), aho[u].pch);
        }
    }
    return aho[u].link;
}
int go(int u, char ch) {
    int c = ch - 'a';
    if (aho[u].go[c] == -1) {
        if (aho[u].next[c] != -1) {
            aho[u].go[c] = aho[u].next[c];
        } else {
            aho[u].go[c] = u == 0 ? 0 : go(get_link(u), ch);
        }
    }
    return aho[u].go[c];
}
int exi(int u) {
    if (aho[u].exi != -1) {
        return aho[u].exi;
    }
    int v = get_link(u);
    return aho[u].exi = (v == 0 || aho[v].term ? v : exi(v));
}
void process(const string &s) {
    int st = 0;
    for (char c : s) {
        st = go(st, c);
        for (int aux = st; aux; aux = exi(aux)) {
            aho[aux].cont++;
        }
    }
    for (int st = 1; st < aho_sz; st++) {
        if (!aho[st].term) {
            continue;
        }
        for (int i : aho[st].idxs) {
            // Do something here
            // idx i ocurs + aho[st].cont times
            h[i] += aho[st].cont;
        }
    }
}
\end{lstlisting}
\end{multicols}

\hfill

\section{Trie}


Estrutura que guarda informações indexadas por palavra.   



Útil encontrar todos os prefixos inseridos anteriormente de uma palavra específica.



* Complexidade de tempo (Update): O(|S|)

* Complexidade de tempo (Consulta de palavra): O(|S|)

\hfill

\begin{multicols}{2}
\begin{lstlisting}[language=C++]
struct trie {
    map<char, int> trie[100005];
    int value[100005];
    int n_nodes = 0;
    void insert(string &s, int v) {
        int id = 0;
        for (char c : s) {
            if (!trie[id].count(c)) {
                trie[id][c] = ++n_nodes;
            }
            id = trie[id][c];
        }
        value[id] = v;
    }
    int get_value(string &s) {
        int id = 0;
        for (char c : s) {
            if (!trie[id].count(c)) {
                return -1;
            }
            id = trie[id][c];
        }
        return value[id];
    }
};
\end{lstlisting}
\end{multicols}

\hfill

\section{Suffix Array}


Estrutura que conterá inteiros que representam os índices iniciais de todos os sufixos ordenados de uma determinada string.



Tambem Constroi a tabela LCP(Longest common prefix).



* Complexidade de tempo (Pré-Processamento): O(|S|*log(|S|))

* Complexidade de tempo (Contar ocorrencias de S em T): O(|S|*log(|T|))
\hfill

\begin{multicols}{2}
\begin{lstlisting}[language=C++]
pair<int, int> busca(string &t, int i, pair<int, int> &range) {
    int esq = range.first, dir = range.second, L = -1, R = -1;
    while (esq <= dir) {
        int mid = (esq + dir) / 2;
        if (s[sa[mid] + i] == t[i]) {
            L = mid;
        }
        if (s[sa[mid] + i] < t[i]) {
            esq = mid + 1;
        } else {
            dir = mid - 1;
        }
    }
    esq = range.first, dir = range.second;
    while (esq <= dir) {
        int mid = (esq + dir) / 2;
        if (s[sa[mid] + i] == t[i]) {
            R = mid;
        }
        if (s[sa[mid] + i] <= t[i]) {
            esq = mid + 1;
        } else {
            dir = mid - 1;
        }
    }
    return {L, R};
}
// count ocurences of s on t
int busca_string(string &t) {
    pair<int, int> range = {0, n - 1};
    for (int i = 0; i < t.size(); i++) {
        range = busca(t, i, range);
        if (range.first == -1) {
            return 0;
        }
    }
    return range.second - range.first + 1;
}\end{lstlisting}
\end{multicols}

\hfill

\begin{multicols}{2}
\begin{lstlisting}[language=C++]
const int MAX_N = 5e5 + 5;

struct suffix_array {
    string s;
    int n, sum, r, ra[MAX_N], sa[MAX_N], auxra[MAX_N], auxsa[MAX_N], c[MAX_N], lcp[MAX_N];
    void counting_sort(int k) {
        memset(c, 0, sizeof(c));
        for (int i = 0; i < n; i++) {
            c[(i + k < n) ? ra[i + k] : 0]++;
        }
        for (int i = sum = 0; i < max(256, n); i++) {
            sum += c[i], c[i] = sum - c[i];
        }
        for (int i = 0; i < n; i++) {
            auxsa[c[sa[i] + k < n ? ra[sa[i] + k] : 0]++] = sa[i];
        }
        for (int i = 0; i < n; i++) {
            sa[i] = auxsa[i];
        }
    }
    void build_sa() {
        for (int k = 1; k < n; k <<= 1) {
            counting_sort(k);
            counting_sort(0);
            auxra[sa[0]] = r = 0;
            for (int i = 1; i < n; i++) {
                auxra[sa[i]] =
                    (ra[sa[i]] == ra[sa[i - 1]] && ra[sa[i] + k] == ra[sa[i - 1] + k])
                        ? r
                        : ++r;
            }
            for (int i = 0; i < n; i++) {
                ra[i] = auxra[i];
            }
            if (ra[sa[n - 1]] == n - 1) {
                break;
            }
        }
    }
    void build_lcp() {
        for (int i = 0, k = 0; i < n - 1; i++) {
            int j = sa[ra[i] - 1];
            while (s[i + k] == s[j + k]) {
                k++;
            }
            lcp[ra[i]] = k;
            if (k) {
                k--;
            }
        }
    }
    void set_string(string _s) {
        s = _s + '$';
        n = s.size();
        for (int i = 0; i < n; i++) {
            ra[i] = s[i], sa[i] = i;
        }
        build_sa();
        build_lcp();
        // for (int i = 0; i < n; i++)
        // printf("%2d: %s\n", sa[i], s.c_str() +
        // sa[i]);
    }
    int operator[](int i) {
        return sa[i];
    }
} sa;
\end{lstlisting}
\end{multicols}

\hfill

\section{Manacher}


Encontra todos os palindromos de uma string.



Dada uma string s de tamanho n, encontra todos os pares (i,j) tal que a substring $s[i...j]$ seja um palindromo.



* Complexidade de tempo: O(N)
\hfill

\begin{multicols}{2}
\begin{lstlisting}[language=C++]
struct manacher {
    long long n, count;
    vector<int> d1, d2;
    long long solve(string &s) {
        n = s.size(), count = 0;
        solve_odd(s);
        solve_even(s);
        return count;
    }
    void solve_odd(string &s) {
        d1.resize(n);
        for (int i = 0, l = 0, r = -1; i < n; i++) {
            int k = (i > r) ? 1 : min(d1[l + r - i], r - i + 1);
            while (0 <= i - k && i + k < n && s[i - k] == s[i + k]) {
                k++;
            }
            count += d1[i] = k--;
            if (i + k > r) {
                l = i - k;
                r = i + k;
            }
        }
    }
    void solve_even(string &s) {
        d2.resize(n);
        for (int i = 0, l = 0, r = -1; i < n; i++) {
            int k = (i > r) ? 0 : min(d2[l + r - i + 1], r - i + 1);
            while (0 <= i - k - 1 && i + k < n && s[i - k - 1] == s[i + k]) {
                k++;
            }
            count += d2[i] = k--;
            if (i + k > r) {
                l = i - k - 1;
                r = i + k;
            }
        }
    }
} mana;
\end{lstlisting}
\end{multicols}

\hfill

\section{Patricia Tree}


Estrutura de dados que armazena strings e permite consultas por prefixo.



Implementação PB-DS, extremamente curta e confusa:



\begin{itemize}
\item Criar: \lstinline{patricia_tree pat;}
\item Inserir: \lstinline{pat.insert("sei la");}
\item Remover: \lstinline{pat.erase("sei la");}
\item Verificar existência: \lstinline{pat.find("sei la") != pat.end();}
\item Pegar palavras que começam com um prefixo: \lstinline{auto match = pat.prefix_range("sei");}
\item Percorrer *match* : \lstinline{for(auto it = match.first; it != match.second; ++it);}
\item Pegar menor elemento lexicográfico *maior ou igual* ao prefixo: \lstinline{*pat.lower_bound("sei");}
\item Pegar menor elemento lexicográfico *maior* ao prefixo: \lstinline{*pat.upper_bound("sei");}
\end{itemize}



\textbf{TODAS AS OPERAÇÕES EM O( |S| )}  

\textbf{NÃO ACEITA ELEMENTOS REPETIDOS}

\hfill

\begin{multicols}{2}
\begin{lstlisting}[language=C++]
#include <ext/pb_ds/assoc_container.hpp>
#include <ext/pb_ds/trie_policy.hpp>

using namespace __gnu_pbds;
typedef trie<string, null_type, trie_string_access_traits<>, pat_trie_tag,
             trie_prefix_search_node_update>
    patricia_tree;
\end{lstlisting}
\end{multicols}

\hfill

\section{Prefix Function}


Para cada prefixo k de uma dada string s, calcula o maior prefixo que tambem é sufixo de k.



Seja n o tamanho do texto e m o tamanho do padrão.



\textbf{KMP} 



String matching em O(n + m).



\textbf{Autômato de KMP} 



String matching em O(n) com O(m) de pré-processamento.



\textbf{Prefix Count} 



Dada uma string s, calcula quantas vezes cada prefixo de s aparece em s com complexidade de tempo de O(n).

\hfill

\begin{multicols}{2}
\begin{lstlisting}[language=C++]
vector<int> pi(string &s) {
    vector<int> p(s.size());
    for (int i = 1, j = 0; i < s.size(); i++) {
        while (j > 0 && s[i] != s[j]) {
            j = p[j - 1];
        }
        if (s[i] == s[j]) {
            j++;
        }
        p[i] = j;
    }
    return p;
}
\end{lstlisting}
\end{multicols}

\hfill

\begin{multicols}{2}
\begin{lstlisting}[language=C++]
vector<int> pi(string &s) {
    vector<int> p(s.size());
    for (int i = 1, j = 0; i < s.size(); i++) {
        while (j > 0 && s[i] != s[j]) {
            j = p[j - 1];
        }
        if (s[i] == s[j]) {
            j++;
        }
        p[i] = j;
    }
    return p;
}

vector<int> kmp(string &s, string t) {
    t += '$';
    vector<int> p = pi(t), match;
    for (int i = 0, j = 0; i < s.size(); i++) {
        while (j > 0 && s[i] != t[j]) {
            j = p[j - 1];
        }
        if (s[i] == t[j]) {
            j++;
        }
        if (j == t.size() - 1) {
            match.push_back(i - j + 1);
        }
    }
    return match;
}
\end{lstlisting}
\end{multicols}

\hfill

\begin{multicols}{2}
\begin{lstlisting}[language=C++]
vector<int> pi(string s) {
    vector<int> p(s.size());
    for (int i = 1, j = 0; i < s.size(); i++) {
        while (j > 0 && s[i] != s[j]) {
            j = p[j - 1];
        }
        if (s[i] == s[j]) {
            j++;
        }
        p[i] = j;
    }
    return p;
}

vector<int> prefixCount(string s) {
    vector<int> p = pi(s + '#');
    int n = s.size();
    vector<int> cnt(n + 1, 0);
    for (int i = 0; i < n; i++) {
        cnt[p[i]]++;
    }
    for (int i = n - 1; i > 0; i--) {
        cnt[p[i - 1]] += cnt[i];
    }
    for (int i = 0; i <= n; i++) {
        cnt[i]++;
    }
    return cnt;
}\end{lstlisting}
\end{multicols}

\hfill

\begin{multicols}{2}
\begin{lstlisting}[language=C++]
struct AutKMP {
    vector<vector<int>> nxt;

    vector<int> pi(string &s) {
        vector<int> p(s.size());
        for (int i = 1, j = 0; i < s.size(); i++) {
            while (j > 0 && s[i] != s[j]) {
                j = p[j - 1];
            }
            if (s[i] == s[j]) {
                j++;
            }
            p[i] = j;
        }
        return p;
    }

    void setString(string s) {
        s += '#';
        nxt.assign(s.size(), vector<int>(26));
        vector<int> p = pi(s);
        for (int c = 0; c < 26; c++) {
            nxt[0][c] = ('a' + c == s[0]);
        }
        for (int i = 1; i < s.size(); i++) {
            for (int c = 0; c < 26; c++) {
                nxt[i][c] = ('a' + c == s[i]) ? i + 1 : nxt[p[i - 1]][c];
            }
        }
    }

    vector<int> kmp(string &s, string &t) {
        vector<int> match;
        for (int i = 0, j = 0; i < s.size(); i++) {
            j = nxt[j][s[i] - 'a'];
            if (j == t.size()) {
                match.push_back(i - j + 1);
            }
        }
        return match;
    }
} aut;
\end{lstlisting}
\end{multicols}

\hfill

\section{Hashing}


Hashing para testar igualdade de duas strings.



A função \textbf{*range(i, j)}* retorna o hash da substring nesse range.   

Pode ser necessário usar pares de hash para evitar colisões.



* Complexidade de tempo (Construção): O(N)

* Complexidade de tempo (Consulta de range): O(1)

\hfill

\begin{multicols}{2}
\begin{lstlisting}[language=C++]
struct hashing {
    const long long LIM = 1000006;
    long long p, m;
    vector<long long> pw, hsh;
    hashing(long long _p, long long _m) : p(_p), m(_m) {
        pw.resize(LIM);
        hsh.resize(LIM);
        pw[0] = 1;
        for (int i = 1; i < LIM; i++) {
            pw[i] = (pw[i - 1] * p) % m;
        }
    }
    void set_string(string &s) {
        hsh[0] = s[0];
        for (int i = 1; i < s.size(); i++) {
            hsh[i] = (hsh[i - 1] * p + s[i]) % m;
        }
    }
    long long range(int esq, int dir) {
        long long ans = hsh[dir];
        if (esq > 0) {
            ans = (ans - (hsh[esq - 1] * pw[dir - esq + 1] % m) + m) % m;
        }
        return ans;
    }
};
\end{lstlisting}
\end{multicols}

\hfill

\section{Lyndon}


Strings em decomposição única em subcadeias que são ordenadas lexicograficamente e não podem ser mais reduzidas.



\textbf{Duval} 



Gera a Lyndon Factorization de uma string



* Complexidade de tempo: O(N)



\textbf{Min Cyclic Shift} 



Gera a menor rotação circular da string original que pode ser obtida por meio de deslocamentos cíclicos dos caracteres.



* Complexidade de tempo: O(N)



\hfill

\begin{multicols}{2}
\begin{lstlisting}[language=C++]
string min_cyclic_shift(string s) {
    s += s;
    int n = s.size();
    int i = 0, ans = 0;
    while (i < n / 2) {
        ans = i;
        int j = i + 1, k = i;
        while (j < n && s[k] <= s[j]) {
            if (s[k] < s[j]) {
                k = i;
            } else {
                k++;
            }
            j++;
        }
        while (i <= k) {
            i += j - k;
        }
    }
    return s.substr(ans, n / 2);
}
\end{lstlisting}
\end{multicols}

\hfill

\begin{multicols}{2}
\begin{lstlisting}[language=C++]
vector<string> duval(string const &s) {
    int n = s.size();
    int i = 0;
    vector<string> factorization;
    while (i < n) {
        int j = i + 1, k = i;
        while (j < n && s[k] <= s[j]) {
            if (s[k] < s[j]) {
                k = i;
            } else {
                k++;
            }
            j++;
        }
        while (i <= k) {
            factorization.push_back(s.substr(i, j - k));
            i += j - k;
        }
    }
    return factorization;
}
\end{lstlisting}
\end{multicols}

\hfill

\newpage

%%%%%%
%
%
% Paradigmas
%
%
%%%%%%

\chapter{Paradigmas}

\section{Busca Ternaria}


Encontra um ponto ótimo em uma função que pode ser separada em duas funções estritamente monotônicas (e.g. parábolas).



\begin{itemize}
\item Complexidade de tempo: O(log(N) * O(eval)). Onde N é o tamanho do espaço de busca e O(eval) o custo de avaliação da função.
\end{itemize}



\textbf{Busca Ternária em Espaço Discreto} 



Encontra um ponto ótimo em uma função que pode ser separada em duas funções estritamente monotônicas (e.g. parábolas).

Versão para espaços discretos.



\begin{itemize}
\item Complexidade de tempo: O(log(N) * O(eval)). Onde N é o tamanho do espaço de busca e O(eval) o custo de avaliação da função.
\end{itemize}

\hfill

\begin{multicols}{2}
\begin{lstlisting}[language=C++]

long long eval(long long mid) {
    // implement the evaluation
}

long long discrete_ternary_search(long long l, long long r) {
    long long ans = -1;
    r--; // to not space r
    while (l <= r) {
        long long mid = (l + r) / 2;

        // minimizing. To maximize use >= to
        // compare
        if (eval(mid) <= eval(mid + 1)) {
            ans = mid;
            r = mid - 1;
        } else {
            l = mid + 1;
        }
    }
    return ans;
}
\end{lstlisting}
\end{multicols}

\hfill

\begin{multicols}{2}
\begin{lstlisting}[language=C++]

double eval(double mid) {
    // implement the evaluation
}

double ternary_search(double l, double r) {
    int k = 100;
    while (k--) {
        double step = (l + r) / 3;
        double mid_1 = l + step;
        double mid_2 = r - step;

        // minimizing. To maximize use >= to
        // compare
        if (eval(mid_1) <= eval(mid_2)) {
            r = mid_2;
        } else {
            l = mid_1;
        }
    }
    return l;
}
\end{lstlisting}
\end{multicols}

\hfill

\section{Convex Hull Trick}


Otimização de DP onde se mantém as retas que formam um Convex Hull em uma estrutura que permite consultar qual o melhor valor para um determinado x.



Só funciona quando as retas são monotônicas. Caso não forem, usar LiChao Tree para guardar as retas



Complexidade de tempo:



\begin{itemize}
\item Inserir reta: O(1) amortizado
\item Consultar x: O(log(N))
\item Consultar x quando x tem crescimento monotônico: O(1)
\end{itemize}

\hfill

\begin{multicols}{2}
\begin{lstlisting}[language=C++]
const ll INF = 1e18 + 18;
bool op(ll a, ll b) {
    return a >= b; // either >= or <=
}
struct line {
    ll a, b;
    ll get(ll x) {
        return a * x + b;
    }
    ll intersect(line l) {
        return (l.b - b + a - l.a) / (a - l.a); // rounds up for integer
                                                // only
    }
};
deque<pair<line, ll>> fila;
void add_line(ll a, ll b) {
    line nova = {a, b};
    if (!fila.empty() && fila.back().first.a == a && fila.back().first.b == b) {
        return;
    }
    while (!fila.empty() && op(fila.back().second, nova.intersect(fila.back().first))) {
        fila.pop_back();
    }
    ll x = fila.empty() ? -INF : nova.intersect(fila.back().first);
    fila.emplace_back(nova, x);
}
ll get_binary_search(ll x) {
    int esq = 0, dir = fila.size() - 1, r = -1;
    while (esq <= dir) {
        int mid = (esq + dir) / 2;
        if (op(x, fila[mid].second)) {
            esq = mid + 1;
            r = mid;
        } else {
            dir = mid - 1;
        }
    }
    return fila[r].first.get(x);
}
// O(1), use only when QUERIES are monotonic!
ll get(ll x) {
    while (fila.size() >= 2 && op(x, fila[1].second)) {
        fila.pop_front();
    }
    return fila.front().first.get(x);
}
\end{lstlisting}
\end{multicols}

\hfill

\section{Busca Binaria Paralela}


Faz a busca binária para múltiplas consultas quando a busca binária é muito pesada.



\begin{itemize}
\item Complexidade de tempo: O((N+Q)log(N) * O(F)), onde N é o tamanho do espaço de busca, Q é o número de consultas e O(F), o custo de avaliação da função.
\end{itemize}

\hfill

\begin{multicols}{2}
\begin{lstlisting}[language=C++]

namespace parallel_binary_search {
    typedef tuple<int, int, long long, long long> query; //{value, id, l, r}
    vector<query> queries[1123456];                      // pode ser um mapa se
                                                         // for muito esparso
    long long ans[1123456];                              // definir pro tamanho
                                                         // das queries
    long long l, r, mid;
    int id = 0;
    void set_lim_search(long long n) {
        l = 0;
        r = n;
        mid = (l + r) / 2;
    }

    void add_query(long long v) {
        queries[mid].push_back({v, id++, l, r});
    }

    void advance_search(long long v) {
        // advance search
    }

    bool satisfies(long long mid, int v, long long l, long long r) {
        // implement the evaluation
    }

    bool get_ans() {
        // implement the get ans
    }

    void parallel_binary_search(long long l, long long r) {

        bool go = 1;
        while (go) {
            go = 0;
            int i = 0; // outra logica se for usar
                       // um mapa
            for (auto &vec : queries) {
                advance_search(i++);
                for (auto q : vec) {
                    auto [v, id, l, r] = q;
                    if (l > r) {
                        continue;
                    }
                    go = 1;
                    // return while satisfies
                    if (satisfies(i, v, l, r)) {
                        ans[i] = get_ans();
                        long long mid = (i + l) / 2;
                        queries[mid] = query(v, id, l, i - 1);
                    } else {
                        long long mid = (i + r) / 2;
                        queries[mid] = query(v, id, i + 1, r);
                    }
                }
                vec.clear();
            }
        }
    }

} // namespace name
\end{lstlisting}
\end{multicols}

\hfill

\section{All Submasks}


Percorre todas as submáscaras de uma máscara.



* Complexidade de tempo: $O(3^N)$

\hfill

\begin{multicols}{2}
\begin{lstlisting}[language=C++]
int mask;
for (int sub = mask; sub; sub = (sub - 1) & mask) { }
\end{lstlisting}
\end{multicols}

\hfill

\section{Divide and Conquer}


Otimização para DP de prefixo quando se pretende separar o vetor em K subgrupos.    



É preciso fazer a função query(i, j) que computa o custo do subgrupo \[i, j\].

* Complexidade de tempo: O(n * k * log(n) *  O(query))



\textbf{Divide and Conquer com Query on demand} 



<!-- *Read in [English](README.en.md)* -->



Usado para evitar queries pesadas ou o custo de pré-processamento.  

É preciso fazer as funções da estrutura \textbf{janela}, eles adicionam e removem itens um a um como uma janela flutuante.



* Complexidade de tempo: O(n * k * log(n) * O(update da janela))





\hfill

\begin{multicols}{2}
\begin{lstlisting}[language=C++]
namespace DC {
    vi dp_before, dp_cur;
    void compute(int l, int r, int optl, int optr) {
        if (l > r) {
            return;
        }
        int mid = (l + r) >> 1;
        pair<ll, int> best = {0, -1}; // {INF, -1} se quiser minimizar
        for (int i = optl; i <= min(mid, optr); i++) {
            best = max(best,
                       {(i ? dp_before[i - 1] : 0) + query(i, mid),
                        i}); // min() se quiser minimizar
        }
        dp_cur[mid] = best.first;
        int opt = best.second;
        compute(l, mid - 1, optl, opt);
        compute(mid + 1, r, opt, optr);
    }

    ll solve(int n, int k) {
        dp_before.assign(n + 5, 0);
        dp_cur.assign(n + 5, 0);
        for (int i = 0; i < n; i++) {
            dp_before[i] = query(0, i);
        }
        for (int i = 1; i < k; i++) {
            compute(0, n - 1, 0, n - 1);
            dp_before = dp_cur;
        }
        return dp_before[n - 1];
    }
};
\end{lstlisting}
\end{multicols}

\hfill

\begin{multicols}{2}
\begin{lstlisting}[language=C++]
namespace DC {
    struct range { // eh preciso definir a forma
                   // de calcular o range
        vi freq;
        ll sum = 0;
        int l = 0, r = -1;
        void back_l(int v) { // Mover o 'l' do range
                             // para a esquerda
            sum += freq[v];
            freq[v]++;
            l--;
        }
        void advance_r(int v) { // Mover o 'r' do range
                                // para a direita
            sum += freq[v];
            freq[v]++;
            r++;
        }
        void advance_l(int v) { // Mover o 'l' do range
                                // para a direita
            freq[v]--;
            sum -= freq[v];
            l++;
        }
        void back_r(int v) { // Mover o 'r' do range
                             // para a esquerda
            freq[v]--;
            sum -= freq[v];
            r--;
        }
        void clear(int n) { // Limpar range
            l = 0;
            r = -1;
            sum = 0;
            freq.assign(n + 5, 0);
        }
    } s;

    vi dp_before, dp_cur;
    void compute(int l, int r, int optl, int optr) {
        if (l > r) {
            return;
        }
        int mid = (l + r) >> 1;
        pair<ll, int> best = {0, -1}; // {INF, -1} se quiser minimizar

        while (s.l < optl) {
            s.advance_l(v[s.l]);
        }
        while (s.l > optl) {
            s.back_l(v[s.l - 1]);
        }
        while (s.r < mid) {
            s.advance_r(v[s.r + 1]);
        }
        while (s.r > mid) {
            s.back_r(v[s.r]);
        }

        vi removed;
        for (int i = optl; i <= min(mid, optr); i++) {
            best =
                min(best,
                    {(i ? dp_before[i - 1] : 0) + s.sum, i}); // min() se quiser minimizar
            removed.push_back(v[s.l]);
            s.advance_l(v[s.l]);
        }
        for (int rem : removed) {
            s.back_l(v[s.l - 1]);
        }

        dp_cur[mid] = best.first;
        int opt = best.second;
        compute(l, mid - 1, optl, opt);
        compute(mid + 1, r, opt, optr);
    }

    ll solve(int n, int k) {
        dp_before.assign(n, 0);
        dp_cur.assign(n, 0);
        s.clear(n);
        for (int i = 0; i < n; i++) {
            s.advance_r(v[i]);
            dp_before[i] = s.sum;
        }
        for (int i = 1; i < k; i++) {
            s.clear(n);
            compute(0, n - 1, 0, n - 1);
            dp_before = dp_cur;
        }
        return dp_before[n - 1];
    }
};
\end{lstlisting}
\end{multicols}

\hfill

\section{Exponenciação de Matriz}


Otimização para DP de prefixo quando o valor atual está em função dos últimos $K$ valores já calculados.   



* Complexidade de tempo: $O(log(n)*k^3)$



É preciso mapear a DP para uma exponenciação de matriz.



\textbf{} 


DP:   



$$ dp[n] = \sum_{i=1}^{k} c[i] \cdot dp[n - i] $$ 



Mapeamento:   



$$ \begin{pmatrix} 0&1&0&0&...&0 \\\ 0&0&1&0&...&0 \\\ 0&0&0&1&...&0 \\\ ...&...&...&...&...&... \\\ c[k]&c[k-1]&c[k-2]&...&c[1]&0 \end{pmatrix}^n \times \begin{pmatrix} dp[0] \\\ dp[1] \\\ dp[2] \\\ ... \\\ dp[k-1] \end{pmatrix} $$



\begin{itemize}
\item --
\end{itemize}

\textbf{} 


Exemplo de DP:   



$$ dp[i] = dp[i-1] + 2 \cdot i^2 + 3 \cdot i + 5 $$



Nesses casos é preciso fazer uma linha para manter cada constante e potência do índice.



Mapeamento:



$$ \begin{pmatrix} 1&5&3&2 \\\ 0&1&0&0 \\\ 0&1&1&0 \\\ 0&1&2&1 \end{pmatrix}^n \times \begin{pmatrix} dp[0]   \\\ 1       \\\ 1       \\\ 1\end {pmatrix} \begin{matrix} \text{mantém } dp[i]   \\\ \text{mantém }1       \\\ \text{mantém }i       \\\ \text{mantém }i^2      \end {matrix} $$



\textbf{} 


Exemplo de DP:



$$ dp[n] =  c\times \prod_{i=1}^{k} dp[n-i] $$



Nesses casos é preciso trabalhar com o logaritmo e temos o caso padrão:





$$ \log(dp[n]) =  log(c) + \sum_{i=1}^{k} log(dp[n-i]) $$



Se a resposta precisar ser inteira, deve-se fatorar a constante e os valores inicias e então fazer uma exponenciação para cada fator primo. Depois é só juntar a resposta no final.

\hfill

\begin{multicols}{2}
\begin{lstlisting}[language=C++]
ll dp[100];
mat T;

#define MOD 1000000007

mat mult(mat a, mat b) {
    mat res(a.size(), vi(b[0].size()));
    for (int i = 0; i < a.size(); i++) {
        for (int j = 0; j < b[0].size(); j++) {
            for (int k = 0; k < b.size(); k++) {
                res[i][j] += a[i][k] * b[k][j] % MOD;
                res[i][j] %= MOD;
            }
        }
    }
    return res;
}

mat exp_mod(mat b, ll exp) {
    mat res(b.size(), vi(b.size()));
    for (int i = 0; i < b.size(); i++) {
        res[i][i] = 1;
    }

    while (exp) {
        if (exp & 1) {
            res = mult(res, b);
        }
        b = mult(b, b);
        exp /= 2;
    }
    return res;
}

// MUDA MUITO DE ACORDO COM O PROBLEMA
// LEIA COMO FAZER O MAPEAMENTO NO README
ll solve(ll exp, ll dim) {
    if (exp < dim) {
        return dp[exp];
    }

    T.assign(dim, vi(dim));
    // TO DO: Preencher a Matriz que vai ser
    // exponenciada T[0][1] = 1; T[1][0] = 1;
    // T[1][1] = 1;

    mat prod = exp_mod(T, exp);

    mat vec;
    vec.assign(dim, vi(1));
    for (int i = 0; i < dim; i++) {
        vec[i][0] = dp[i]; // Valores iniciais
    }

    mat ans = mult(prod, vec);
    return ans[0][0];
}
\end{lstlisting}
\end{multicols}

\hfill

\section{DP de Permutacao}


Otimização do problema do Caixeiro Viajante



* Complexidade de tempo: $O(n^2 * 2^n)$



Para rodar a função basta setar a matriz de adjacência 'dist' e chamar solve(0,0,n).

\hfill

\begin{multicols}{2}
\begin{lstlisting}[language=C++]
const int lim = 17;         // setar para o maximo de itens
long double dist[lim][lim]; // eh preciso dar as
                            // distancias de n para n
long double dp[lim][1 << lim];

int limMask = (1 << lim) - 1; // 2**(maximo de itens) - 1
long double solve(int atual, int mask, int n) {
    if (dp[atual][mask] != 0) {
        return dp[atual][mask];
    }
    if (mask == (1 << n) - 1) {
        return dp[atual][mask] = 0; // o que fazer quando
                                    // chega no final
    }

    long double res = 1e13; // pode ser maior se precisar
    for (int i = 0; i < n; i++) {
        if (!(mask & (1 << i))) {
            long double aux = solve(i, mask | (1 << i), n);
            if (mask) {
                aux += dist[atual][i];
            }
            res = min(res, aux);
        }
    }
    return dp[atual][mask] = res;
}
\end{lstlisting}
\end{multicols}

\hfill

\section{Mo}


Resolve Queries Complicadas Offline de forma rápida.  



É preciso manter uma estrutura que adicione e remova elementos nas extremeidades de um range (tipo janela).



\begin{itemize}
\item Complexidade de tempo (Query offline): O(N * sqrt(N))
\end{itemize}



\textbf{Mo com Update} 



Resolve Queries Complicadas Offline de forma rápida.   

Permite que existam \textbf{UPDATES PONTUAIS!}  

É preciso manter uma estrutura que adicione e remova elementos nas extremidades de um range (tipo janela).



\begin{itemize}
\item Complexidade de tempo: $O(Q * N^(2/3))$
\end{itemize}

\hfill

\begin{multicols}{2}
\begin{lstlisting}[language=C++]
typedef pair<int, int> ii;
int block_sz; // Better if 'const';

namespace mo {
    struct query {
        int l, r, idx;
        bool operator<(query q) const {
            int _l = l / block_sz;
            int _ql = q.l / block_sz;
            return ii(_l, (_l & 1 ? -r : r)) < ii(_ql, (_ql & 1 ? -q.r : q.r));
        }
    };
    vector<query> queries;

    void build(int n) {
        block_sz = (int)sqrt(n);
        // TODO: initialize data structure
    }
    inline void add_query(int l, int r) {
        queries.push_back({l, r, (int)queries.size()});
    }
    inline void remove(int idx) {
        // TODO: remove value at idx from data
        // structure
    }
    inline void add(int idx) {
        // TODO: add value at idx from data
        // structure
    }
    inline int get_answer() {
        // TODO: extract the current answer of the
        // data structure
        return 0;
    }

    vector<int> run() {
        vector<int> answers(queries.size());
        sort(queries.begin(), queries.end());
        int L = 0;
        int R = -1;
        for (query q : queries) {
            while (L > q.l) {
                add(--L);
            }
            while (R < q.r) {
                add(++R);
            }
            while (L < q.l) {
                remove(L++);
            }
            while (R > q.r) {
                remove(R--);
            }
            answers[q.idx] = get_answer();
        }
        return answers;
    }

};
\end{lstlisting}
\end{multicols}

\hfill

\begin{multicols}{2}
\begin{lstlisting}[language=C++]
typedef pair<int, int> ii;
typedef tuple<int, int, int> iii;
int block_sz; // Better if 'const';
vector<int> vec;
namespace mo {
    struct query {
        int l, r, t, idx;
        bool operator<(query q) const {
            int _l = l / block_sz;
            int _r = r / block_sz;
            int _ql = q.l / block_sz;
            int _qr = q.r / block_sz;
            return iii(_l, (_l & 1 ? -_r : _r), (_r & 1 ? t : -t)) <
                   iii(_ql, (_ql & 1 ? -_qr : _qr), (_qr & 1 ? q.t : -q.t));
        }
    };
    vector<query> queries;
    vector<ii> updates;

    void build(int n) {
        block_sz = pow(1.4142 * n, 2.0 / 3);
        // TODO: initialize data structure
    }
    inline void add_query(int l, int r) {
        queries.push_back({l, r, (int)updates.size(), (int)queries.size()});
    }
    inline void add_update(int x, int v) {
        updates.push_back({x, v});
    }
    inline void remove(int idx) {
        // TODO: remove value at idx from data
        // structure
    }
    inline void add(int idx) {
        // TODO: add value at idx from data
        // structure
    }
    inline void update(int l, int r, int t) {
        auto &[x, v] = updates[t];
        if (l <= x && x <= r) {
            remove(x);
        }
        swap(vec[x], v);
        if (l <= x && x <= r) {
            add(x);
        }
    }
    inline int get_answer() {
        // TODO: extract the current answer from
        // the data structure
        return 0;
    }

    vector<int> run() {
        vector<int> answers(queries.size());
        sort(queries.begin(), queries.end());
        int L = 0;
        int R = -1;
        int T = 0;
        for (query q : queries) {
            while (T < q.t) {
                update(L, R, T++);
            }
            while (T > q.t) {
                update(L, R, --T);
            }
            while (L > q.l) {
                add(--L);
            }
            while (R < q.r) {
                add(++R);
            }
            while (L < q.l) {
                remove(L++);
            }
            while (R > q.r) {
                remove(R--);
            }
            answers[q.idx] = get_answer();
        }
        return answers;
    }
};
\end{lstlisting}
\end{multicols}

\hfill

\chapter{Theoretical}

\twocolumn

\section{Some Prime Numbers}

\subsection{Left-Truncatable Prime}
\large{Prime number such that any suffix of it is a prime number} \\
\Large{357,686,312,646,216,567,629,137}

\subsection{Mersenne Primes}
\large{Prime numbers of the form $2^m-1$}
\begin{table}[h]
  \centering
  \begin{tabular}{|c|c|}
    \hline
    \textbf{Exponent ($m$)} & \textbf{Decimal representation} \\
    \hline
    \(2\) & 3 \\
    \hline
    \(3\) & 7 \\
    \hline
    \(5\) & 31 \\
    \hline
    \(7\) & 127 \\
    \hline
    \(13\) & 8,191 \\
    \hline
    \(17\) & 131,071 \\
    \hline
    \(19\) & 524,287 \\
    \hline
    \(31\) & 2,147,483,647 \\
    \hline
    \(61\) & $2,3 * 10^{18}$ \\
    \hline
    \(89\) & $6,1 * 10^{26}$ \\
    \hline
    \(107\) & $1,6 * 10^{32}$ \\
    \hline
    \(127\) & $1,7 * 10^{38}$ \\
    \hline
  \end{tabular}
\end{table}

\section{C++ constants}

\begin{center}
\begin{tabular}{|c|c|c|}
  \hline
  Constant & C++ Name & Value \\
  \hline
  $\pi$ & \texttt{M\_PI} & 3.141592... \\
  \hline
  $\pi / 2$ & \texttt{M\_PI\_2} & 1.570796... \\
  \hline
  $\pi / 4$ & \texttt{M\_PI\_4} & 0.785398... \\
  \hline
  $1 / \pi$ & \texttt{M\_1\_PI} & 0.318309... \\
  \hline
  $2 / \pi$ & \texttt{M\_2\_PI} & 0.636619... \\
  \hline
  $2 / \sqrt{\pi}$ & \texttt{M\_2\_SQRTPI} & 1.128379... \\
  \hline
  $\sqrt{2}$ & \texttt{M\_SQRT2} & 1.414213... \\
  \hline
  $1 / \sqrt{2}$ & \texttt{M\_SQRT1\_2} & 0.707106... \\
  \hline
  $e$ & \texttt{M\_E} & 2.718281... \\
  \hline
  $\log_2{e}$ & \texttt{M\_LOG2E} & 1.442695... \\
  \hline
  $\log_{10}{e}$ & \texttt{M\_LOG10E} & 0.434294... \\
  \hline
  $\ln{2}$ & \texttt{M\_LN2} & 0.693147... \\
  \hline
  $\ln{10}$ & \texttt{M\_LN10} & 2.302585... \\
  \hline
\end{tabular}
\end{center}

\section{Linear Operators}

\subsection{Rotate counter-clockwise by $\theta ^\circ$}
\begin{equation*} 
\Large{
\begin{bmatrix}
    cos \theta & -sin \theta \\
    sin \theta & cos \theta
\end{bmatrix}
}
\end{equation*}


\subsection{Reflect about the line $y = mx$ }
\begin{equation*}
\LARGE{\frac{1}{m^2+1}}
\large{
\begin{bmatrix}
    1 - m^2 & 2m \\
    2m & m^2 - 1
\end{bmatrix}
}  
\end{equation*} 

\subsection{Inverse of a 2x2 matrix A}
\begin{equation*} 
\large{
\begin{bmatrix}
    a & b \\
    c & d
\end{bmatrix}^{-1} =
\LARGE{\frac{1}{det(A)}}
\large{
\begin{bmatrix}
    d & -b \\
    -c & a
\end{bmatrix}
}
}
\end{equation*}

\subsection{Horizontal shear by K}
\begin{equation*} 
\Large{
\begin{bmatrix}
    1 & K \\
    0 & 1
\end{bmatrix}
}
\end{equation*}


\subsection{Vertical shear by K}
\begin{equation*} 
\Large{
\begin{bmatrix}
    1 & 0 \\
    K & 1
\end{bmatrix}
}
\end{equation*}

\subsection{Change of basis}
\Large {$\vec{a}_\beta$ are the coordinates of vector $\vec{a}$ in basis $\beta$.\\}
\Large {$\vec{a}$ are the coordinates of vector $\vec{a}$ in the canonical basis.\\}
\Large {$\vec{b1}$ and $\vec{b2}$ are the basis vectors for $\beta$.\\}
\Large {$C$ is a matrix that changes from basis $\beta$ to the canonical basis.}
\begin{equation*}
\Large{C \vec{a}_\beta = \vec{a}}
\end{equation*} 
\begin{equation*}
\Large{C^{-1} \vec{a} = \vec{a}_\beta}
\end{equation*} 
\begin{equation*} 
\Large{
C = 
\begin{bmatrix}
    b1_x & b2_x \\
    b1_y & b2_y
\end{bmatrix}
}
\end{equation*}


\subsection{Properties of matrix operations}
\begin{equation*} 
\large{(AB)^{-1} = A^{-1}B^{-1}}\\
\end{equation*}
\begin{equation*} 
\large{(AB)^{T} = B^{T}A^{T}}\\
\end{equation*}
\begin{equation*} 
\large{(A^{-1})^{T} = (A^{T})^{-1}}\\
\end{equation*}
\begin{equation*} 
\large{(A+B)^{T} = A^{T} + B^{T}}\\
\end{equation*}
\begin{equation*} 
\large{det(A) = det(A^{T})}\\
\end{equation*}
\begin{equation*} 
\large{det(AB) = det(A)det(B)}\\
\end{equation*}
\large{Let $A$ be an NxN matrix:}
\begin{equation*} 
\large{det(kA) = K^Ndet(A)}\\
\end{equation*}
\end{document}
