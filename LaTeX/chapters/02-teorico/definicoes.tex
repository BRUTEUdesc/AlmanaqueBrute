\subsection{Definições}

Algumas definições e termos importantes:

\subsubsection{Funções}

\begin{itemize}
  \item \textbf{Comutativa}: Uma função $f(x, y)$ é comutativa se $f(x, y) = f(y, x)$.
  \item \textbf{Associativa}: Uma função $f(x, y)$ é associativa se $f(x, f(y, z)) = f(f(x, y), z)$.
  \item \textbf{Idempotente}: Uma função $f(x, y)$ é idempotente se $f(x, x) = x$.
\end{itemize}

\subsubsection{Grafos}

\begin{itemize}
  \item \textbf{Grafo}: Um grafo é um conjunto de vértices e um conjunto de arestas que conectam os vértices.
  \item \textbf{Grafo Conexo}: Um grafo é conexo se existe um caminho entre todos os pares de vértices.
  \item \textbf{Grafo Bipartido}: Um grafo é bipartido se é possível dividir os vértices em dois conjuntos disjuntos de forma que todas as arestas conectem um vértice de um conjunto com um vértice do outro conjunto, ou seja, não existem arestas que conectem vértices do mesmo conjunto.
  \item \textbf{Árvore}: Um grafo é uma árvore se ele é conexo e não possui ciclos.
  \item \textbf{Árvore Geradora Mínima (AGM)}: Uma árvore geradora mínima é uma árvore que conecta todos os vértices de um grafo e possui o menor custo possível, também conhecido como \textit{Minimum Spanning Tree (MST)}.
\end{itemize}
