\subsection{Operadores lineares}

\subsubsection{Rotação no sentido anti-horário por $\theta ^\circ$}

\begin{equation*}
\large
\begin{bmatrix}
    \cos \theta & -\sin \theta \\
    \sin \theta & \cos \theta
\end{bmatrix}
\end{equation*}

\subsubsection{Reflexão em relação à reta $y = mx$}

\begin{equation*}
\large
\frac{1}{m^2+1}
\begin{bmatrix}
    1 - m^2 & 2m \\
    2m & m^2 - 1
\end{bmatrix}
\end{equation*}

\subsubsection{Inversa de uma matriz $2 \times 2$ $A$}

\begin{equation*}
\large
\begin{bmatrix}
    a & b \\
    c & d
\end{bmatrix}^{-1} =
\frac{1}{\det(A)}
\begin{bmatrix}
    d & -b \\
    -c & a
\end{bmatrix}
\end{equation*}

\subsubsection{Cisalhamento horizontal por $K$}

\begin{equation*}
\large
\begin{bmatrix}
    1 & K \\
    0 & 1
\end{bmatrix}
\end{equation*}

\subsubsection{Cisalhamento vertical por $K$}

\begin{equation*}
\large
\begin{bmatrix}
    1 & 0 \\
    K & 1
\end{bmatrix}
\end{equation*}

\subsubsection{Mudança de base}

{\large $\vec{a}_\beta$ são as coordenadas do vetor $\vec{a}$ na base $\beta$.\\}
{\large $\vec{a}$ são as coordenadas do vetor $\vec{a}$ na base canônica.\\}
{\large $\vec{b1}$ e $\vec{b2}$ são os vetores de base para $\beta$.\\}
{\large $C$ é uma matriz que muda da base $\beta$ para a base canônica.}

\begin{equation*}
\large C \vec{a}_\beta = \vec{a}
\end{equation*}
\begin{equation*}
\large C^{-1} \vec{a} = \vec{a}_\beta
\end{equation*}
\begin{equation*}
\large
C =
\begin{bmatrix}
    b1_x & b2_x \\
    b1_y & b2_y
\end{bmatrix}
\end{equation*}

\subsubsection{Propriedades das operações de matriz}

\begin{equation*}
\large (AB)^{-1} = A^{-1}B^{-1}\\
\end{equation*}
\begin{equation*}
\large (AB)^{T} = B^{T}A^{T} \\
\end{equation*}
\begin{equation*}
\large (A^{-1})^{T} = (A^{T})^{-1} \\
\end{equation*}
\begin{equation*}
\large (A+B)^{T} = A^{T} + B^{T} \\
\end{equation*}
\begin{equation*}
\large \det(A) = \det(A^{T}) \\
\end{equation*}
\begin{equation*}
\large \det(AB) = \det(A)\det(B) \\
\end{equation*}
{\large Seja $A$ uma matriz $N \times N$:}
\begin{equation*}
\large \det(kA) = k^N\det(A) \\
\end{equation*}
