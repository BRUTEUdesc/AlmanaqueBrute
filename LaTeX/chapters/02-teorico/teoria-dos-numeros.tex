\subsection{Teoria dos números}

\subsubsection{Pequeno teorema de Fermat}

Se $p$ é um número primo e $a$ é um inteiro não divisível por $p$, então $a^{p-1} \equiv 1 \pmod{p}$.

\subsubsection{Teorema de Euler}

Se $m$ e $a$ são inteiros positivos coprimos, então $a^{\phi(m)} \equiv 1 \pmod{m}$, onde $\phi(m)$ é a função totiente de Euler.

\subsubsection{Aritmética modular}

Quando estamos trabalhando com aritmética módulo um número $p$, todos os valores existentes estão entre $[0, p-1]$.

Algumas propriedades e equivalências úteis para usar aritmética modular em código são:

\begin{itemize}
  \item $(a + b) \mod p \equiv ((a \mod p) + (b \mod p)) \mod p$
  \item $(a - b) \mod p \equiv ((a \mod p) - (b \mod p)) \mod p$
  \begin{itemize}
    \item Note que o resultado pode ser negativo, nesse caso é necessário adicionar $p$ ao resultado. De forma geral, geralmente fazemos $(a - b + p) \mod p$ (assumindo que $a$ e $b$ já estão no intervalo $[0, p-1]$).
  \end{itemize}
  \item $(a \cdot b) \mod p \equiv ((a \mod p) \cdot (b \mod p)) \mod p$
  \item $a^b \mod p \equiv ((a \mod p)^b) \mod p$
  \item $a^b \mod p \equiv ((a \mod p)^{(b \mod \phi(p))}) \mod p$ se $a$ e $p$ são coprimos
\end{itemize}
