\subsection{Números primos}

Números primos são muito úteis para funções de hashing (dentre outras coisas). Aqui vão vários números primos interessantes:

\subsubsection{Primos com truncamento à esquerda}

Números primos tais que qualquer sufixo deles é um número primo:

\begin{center}
\LARGE
33,333,31 \\
\LARGE
357,686,312,646,216,567,629,137
\end{center}

\subsubsection{Primos gêmeos (Twin Primes)}

Pares de primos da forma $(p, p + 2)$ (aqui tem só alguns pares aleatórios, existem muitos outros).

\begin{table}[h]
  \centering
  \begin{tabular}{|c|c|c|}
    \hline
    \textbf{Primo} & \textbf{Primo + 2} & \textbf{Ordem} \\
    \hline
    5 & 7 & $10^0$ \\
    \hline
    17 & 19 & $10^1$ \\
    \hline
    461 & 463 & $10^2$ \\
    \hline
    3461 & 3463 & $10^3$ \\
    \hline
    34499 & 34501 & $10^4$ \\
    \hline
    487829 & 487831 & $10^5$ \\
    \hline
    5111999 & 5112001 & $10^6$ \\
    \hline
    30684887 & 30684889 & $10^7$ \\
    \hline
    361290539 & 361290541 & $10^8$ \\
    \hline
    1000000007 & 1000000009 & $10^9$ \\
    \hline
    1005599459 & 1005599461 & $10^9$ \\
    \hline
  \end{tabular}
\end{table}

\subsubsection{Números primos de Mersenne}

São os números primos da forma $2^m - 1$, onde $m$ é um número inteiro positivo.

\begin{table}[h]
  \centering
  \begin{tabular}{|c|c|}
    \hline
    \textbf{Expoente ($m$)} & \textbf{Representação Decimal} \\
    \hline
    \(2\) & 3 \\
    \hline
    \(3\) & 7 \\
    \hline
    \(5\) & 31 \\
    \hline
    \(7\) & 127 \\
    \hline
    \(13\) & 8,191 \\
    \hline
    \(17\) & 131,071 \\
    \hline
    \(19\) & 524,287 \\
    \hline
    \(31\) & 2,147,483,647 \\
    \hline
    \(61\) & $2,3 * 10^{18}$ \\
    \hline
    \(89\) & $6,1 * 10^{26}$ \\
    \hline
    \(107\) & $1,6 * 10^{32}$ \\
    \hline
    \(127\) & $1,7 * 10^{38}$ \\
    \hline
  \end{tabular}
\end{table}
