\subsection{Análise combinatória}

\subsubsection{Fatorial}

O fatorial de um número $n$ é o produto de todos os inteiros positivos menores ou iguais a $n$.

O fatorial conta o número de permutações de $n$ elementos.

\begin{equation*}
\large
n! = n \cdot (n-1)!
\end{equation*}

Em particular, $0! = 1$.

\subsubsection{Combinação}

Conta o número de maneiras de escolher $k$ elementos de um conjunto de $n$ elementos.

\begin{equation*}
\large
\binom{n}{k} = \frac{n!}{k! \cdot (n-k)!}
\end{equation*}

\subsubsection{Arranjo}

Conta o número de maneiras de escolher $k$ elementos de um conjunto de $n$ elementos, onde a ordem importa.

\begin{equation*}
\large
P(n, k) = \frac{n!}{(n-k)!}
\end{equation*}

\subsubsection{Estrelas e barras}

Conta o número de maneiras de distribuir $n$ elementos idênticos em $k$ recipientes distintos.

\begin{equation*}
\large
\binom{n+k-1}{k-1}
\end{equation*}

\subsubsection{Princípio da inclusão-exclusão}

O princípio da inclusão-exclusão é uma técnica para contar o número de elementos em uma união de conjuntos.

\begin{align*}
\left| \bigcup_{i=1}^{n} A_i \right|
  &= \sum_{k=1}^{n} (-1)^{k+1}
     \sum_{\substack{1 \leq i_1 < \cdots < i_k \leq n}}
     \left| A_{i_1} \cap \cdots \cap A_{i_k} \right|
\end{align*}

\subsubsection{Princípio da casa dos pombos}

Se $n$ pombos são colocados em $m$ casas, então pelo menos uma casa terá $\lceil \frac{n}{m} \rceil$ pombos ou mais.
