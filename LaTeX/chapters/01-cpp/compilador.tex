\subsection{Compilador}

Para compilar um arquivo \texttt{.cpp} com o compilador \texttt{g++}, usar o comando:

\begin{lstlisting}[language=bash]
    g++ -std=c++20 -O2 arquivo.cpp
\end{lstlisting}

\textbf{Obs:} a flag \texttt{-std=c++20} é para usar a versão 20 do C++, os códigos desse Almanaque são testados com essa versão.

Algumas flags úteis para o \texttt{g++} são:

\begin{itemize}
    \item \texttt{-O2}: Otimizações de compilação
    \item \texttt{-Wall}: Mostra todos os warnings
    \item \texttt{-Wextra}: Mostra mais warnings
    \item \texttt{-Wconversion}: Mostra warnings para conversões implícitas
    \item \texttt{-fsanitize=address}: Habilita o AddressSanitizer
    \item \texttt{-fsanitize=undefined}: Habilita o UndefinedBehaviorSanitizer
\end{itemize}

Todas essas flags já estão presente no script `run` da seção Extra.
