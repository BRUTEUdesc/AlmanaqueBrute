\subsection{Pragmas}

Os pragmas são diretivas para o compilador, que podem ser usadas para otimizar o código.

Temos os pragmas de otimização, como por exemplo:

\begin{itemize}
    \item \texttt{\#pragma GCC optimize("O2")}: Otimizações de nível 2 (padrão de competições)
    \item \texttt{\#pragma GCC optimize("O3")}: Otimizações de nível 3 (seguro para usar)
    \item \texttt{\#pragma GCC optimize("Ofast")}: Otimizações agressivas (perigoso!)
    \item \texttt{\#pragma GCC optimize("unroll-loops")}: Otimiza os loops mas pode levar a \textit{cache misses}
\end{itemize}

E também os pragmas de \textit{target}, que são usados para otimizar o código para um certo processador:

\begin{itemize}
    \item \texttt{\#pragma GCC target("avx2")}: Otimiza instruções para processadores com suporte a AVX2
    \item \texttt{\#pragma GCC target("sse4")}: Parecido com o de cima, mas mais antigo
    \item \texttt{\#pragma GCC target("popcnt")}: Otimiza o \textit{popcount} em processadores que suportam
    \item \texttt{\#pragma GCC target("lzcnt")}: Otimiza o \textit{leading zero count} em processadores que suportam
    \item \texttt{\#pragma GCC target("bmi")}: Otimiza instruções de \textit{bit manipulation} em processadores que suportam
    \item \texttt{\#pragma GCC target("bmi2")}: Mesmo que o de cima, mas mais recente
\end{itemize}

Em geral, esses pragmas são usados para otimizar o código em competições, mas é importante usá-los com certa sabedoria, em alguns casos eles podem piorar o desempenho do código.

Uma opção relativamente segura de se usar é a seguinte:

\begin{lstlisting}[language=C++]
    #pragma GCC optimize("O3,unroll-loops")
    #pragma GCC target("avx2,bmi,bmi2,lzcnt,popcnt")
\end{lstlisting}
