\chapter{Teórico}

\section {Definições}

Algumas definições e termos importantes:

\subsection{Funções}

\begin{itemize}
  \item \textbf{Comutativa}: Uma função $f(x, y)$ é comutativa se $f(x, y) = f(y, x)$.
  \item \textbf{Associativa}: Uma função $f(x, y)$ é associativa se $f(x, f(y, z)) = f(f(x, y), z)$.
  \item \textbf{Idempotente}: Uma função $f(x, y)$ é idempotente se $f(x, x) = x$.
\end{itemize}

\subsection{Grafos}

\begin{itemize}
  \item \textbf{Grafo}: Um grafo é um conjunto de vértices e um conjunto de arestas que conectam os vértices.
  \item \textbf{Grafo Conexo}: Um grafo é conexo se existe um caminho entre todos os pares de vértices.
  \item \textbf{Grafo Bipartido}: Um grafo é bipartido se é possível dividir os vértices em dois conjuntos disjuntos de forma que todas as arestas conectem um vértice de um conjunto com um vértice do outro conjunto, ou seja, não existem arestas que conectem vértices do mesmo conjunto.
  \item \textbf{Árvore}: Um grafo é uma árvore se ele é conexo e não possui ciclos.
  \item \textbf{Árvore Geradora Mínima (AGM)}: Uma árvore geradora mínima é uma árvore que conecta todos os vértices de um grafo e possui o menor custo possível, também conhecido como \textit{Minimum Spanning Tree (MST)}.
\end{itemize}

\section{Números primos}

Números primos são muito úteis para funções de hashing (dentre outras coisas). Aqui vão vários números primos interessantes:

\subsection{Primo com truncamento à esquerda}

\large{Número primo tal que qualquer sufixo dele é um número primo}

\begin{center}
\LARGE{357,686,312,646,216,567,629,137}
\end{center}

\subsection{Primos gêmeos (Twin Primes)}

Pares de primos da forma $(p, p + 2)$ (aqui tem só alguns pares aleatórios, existem muitos outros).

\begin{table}[h]
  \centering
  \begin{tabular}{|c|c|c|}
    \hline
    \textbf{Primo} & \textbf{Primo + 2} & \textbf{Ordem} \\
    \hline
    5 & 7 & $10^0$ \\
    \hline
    17 & 19 & $10^1$ \\
    \hline
    461 & 463 & $10^2$ \\
    \hline
    3461 & 3463 & $10^3$ \\
    \hline
    34499 & 34501 & $10^4$ \\
    \hline
    487829 & 487831 & $10^5$ \\
    \hline
    5111999 & 5112001 & $10^6$ \\
    \hline
    30684887 & 30684889 & $10^7$ \\
    \hline
    361290539 & 361290541 & $10^8$ \\
    \hline
    1000000007 & 1000000009 & $10^9$ \\
    \hline
    1005599459 & 1005599461 & $10^9$ \\
    \hline
  \end{tabular}
\end{table}

\subsection{Números primos de Mersenne}

São os números primos da forma $2^m - 1$, onde $m$ é um número inteiro positivo.

\begin{table}[h]
  \centering
  \begin{tabular}{|c|c|}
    \hline
    \textbf{Expoente ($m$)} & \textbf{Representação Decimal} \\
    \hline
    \(2\) & 3 \\
    \hline
    \(3\) & 7 \\
    \hline
    \(5\) & 31 \\
    \hline
    \(7\) & 127 \\
    \hline
    \(13\) & 8,191 \\
    \hline
    \(17\) & 131,071 \\
    \hline
    \(19\) & 524,287 \\
    \hline
    \(31\) & 2,147,483,647 \\
    \hline
    \(61\) & $2,3 * 10^{18}$ \\
    \hline
    \(89\) & $6,1 * 10^{26}$ \\
    \hline
    \(107\) & $1,6 * 10^{32}$ \\
    \hline
    \(127\) & $1,7 * 10^{38}$ \\
    \hline
  \end{tabular}
\end{table}

\section{Constantes em C++}

\begin{center}
\begin{tabular}{|c|c|c|}
  \hline
  Constante & Nome em \texttt{C++} & Valor \\
  \hline
  $\pi$ & \texttt{M\_PI} & 3.141592... \\
  \hline
  $\pi / 2$ & \texttt{M\_PI\_2} & 1.570796... \\
  \hline
  $\pi / 4$ & \texttt{M\_PI\_4} & 0.785398... \\
  \hline
  $1 / \pi$ & \texttt{M\_1\_PI} & 0.318309... \\
  \hline
  $2 / \pi$ & \texttt{M\_2\_PI} & 0.636619... \\
  \hline
  $2 / \sqrt{\pi}$ & \texttt{M\_2\_SQRTPI} & 1.128379... \\
  \hline
  $\sqrt{2}$ & \texttt{M\_SQRT2} & 1.414213... \\
  \hline
  $1 / \sqrt{2}$ & \texttt{M\_SQRT1\_2} & 0.707106... \\
  \hline
  $e$ & \texttt{M\_E} & 2.718281... \\
  \hline
  $\log_2{e}$ & \texttt{M\_LOG2E} & 1.442695... \\
  \hline
  $\log_{10}{e}$ & \texttt{M\_LOG10E} & 0.434294... \\
  \hline
  $\ln{2}$ & \texttt{M\_LN2} & 0.693147... \\
  \hline
  $\ln{10}$ & \texttt{M\_LN10} & 2.302585... \\
  \hline
\end{tabular}
\end{center}

\section{Operadores lineares}

\subsection{Rotação no sentido anti-horário por $\theta ^\circ$}
\begin{equation*} 
\large{
\begin{bmatrix}
    \cos \theta & -\sin \theta \\
    \sin \theta & \cos \theta
\end{bmatrix}
}
\end{equation*}

\subsection{Reflexão em relação à reta $y = mx$ }
\begin{equation*}
\large{\frac{1}{m^2+1}}
\large{
\begin{bmatrix}
    1 - m^2 & 2m \\
    2m & m^2 - 1
\end{bmatrix}
}  
\end{equation*} 

\subsection{Inversa de uma matriz 2x2 A}
\begin{equation*} 
\large{
\begin{bmatrix}
    a & b \\
    c & d
\end{bmatrix}^{-1} =
\large{\frac{1}{\text{det}(A)}}
\large{
\begin{bmatrix}
    d & -b \\
    -c & a
\end{bmatrix}
}
}
\end{equation*}

\subsection{Cisalhamento horizontal por K}
\begin{equation*} 
\large{
\begin{bmatrix}
    1 & K \\
    0 & 1
\end{bmatrix}
}
\end{equation*}

\subsection{Cisalhamento vertical por K}
\begin{equation*} 
\large{
\begin{bmatrix}
    1 & 0 \\
    K & 1
\end{bmatrix}
}
\end{equation*}

\subsection{Mudança de base}
\large {$\vec{a}_\beta$ são as coordenadas do vetor $\vec{a}$ na base $\beta$.\\}
\large {$\vec{a}$ são as coordenadas do vetor $\vec{a}$ na base canônica.\\}
\large {$\vec{b1}$ e $\vec{b2}$ são os vetores de base para $\beta$.\\}
\large {$C$ é uma matriz que muda da base $\beta$ para a base canônica.}
\begin{equation*}
\large{C \vec{a}_\beta = \vec{a}}
\end{equation*} 
\begin{equation*}
\large{C^{-1} \vec{a} = \vec{a}_\beta}
\end{equation*} 
\begin{equation*} 
\large{
C = 
\begin{bmatrix}
    b1_x & b2_x \\
    b1_y & b2_y
\end{bmatrix}
}
\end{equation*}

\subsection{Propriedades das operações de matriz}
\begin{equation*} 
\large{(AB)^{-1} = A^{-1}B^{-1}}\\
\end{equation*}
\begin{equation*} 
\large{(AB)^{T} = B^{T}A^{T}}\\
\end{equation*}
\begin{equation*} 
\large{(A^{-1})^{T} = (A^{T})^{-1}}\\
\end{equation*}
\begin{equation*} 
\large{(A+B)^{T} = A^{T} + B^{T}}\\
\end{equation*}
\begin{equation*} 
\large{\text{det}(A) = \text{det}(A^{T})}\\
\end{equation*}
\begin{equation*} 
\large{\text{det}(AB) = \text{det}(A)\text{det}(B)}\\
\end{equation*}
\large{Seja $A$ uma matriz NxN:}
\begin{equation*} 
\large{\text{det}(kA) = K^N\text{det}(A)}\\
\end{equation*}